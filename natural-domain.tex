\documentclass[usenames,dvipsnames,fleqn,leqno,10pt, pdflatex]{beamer}
%\usetheme[professionalfonts]{Boadilla}
\usetheme{metropolis}
\setsansfont[BoldFont={Source Sans Pro Semibold}, Numbers={OldStyle}]{Source Sans Pro}
%\usepackage{fontspec}


\metroset{block=fill,background=light}
\setbeamertemplate{blocks}[rounded]

\usepackage{cite}
\usepackage{pgfplots}
%\usepackage[symbol]{footmisc}
%\setmainfont{Liberation Serif}
%\setbeamertemplate{bibliography item}[text]
%\usepackage{graphicx}
\usepackage{bbding}
\setbeamertemplate{caption}[numbered]
\setbeamertemplate{footline}[text line]{%
\parbox{\linewidth}{\vspace*{-8pt}\hfill\insertshortauthor\hfill\insertpagenumber}}
\setbeamertemplate{navigation symbols}{}
\author[bw]{}
\usepackage{amsmath}
\usepackage{amsthm}
\usepackage{upgreek}
\usepackage{isomath}
\usepackage[normalem]{ulem}
%\usepackage{comment}
\usepackage[USenglish]{babel}
\usepackage[activate={true,nocompatibility},final,tracking=true,factor=1100,stretch=10,shrink=10]{microtype}
\usepackage[]{xcolor}
\newcommand{\reals}{\mathbf{R}}
\newcommand{\complex}{\mathbf{C}}
\newcommand{\integers}{\mathbf{Z}}
\newcommand{\imag}{\mathrm{i}}
\DeclareMathOperator{\range}{range}
\DeclareMathOperator{\domain}{domain}
\DeclareMathOperator{\codomain}{codomain}
\DeclareMathOperator{\sspan}{span}
\usepackage{graphicx}

\definecolor{UNK-blue}{HTML}{004d86} 
\definecolor{UNK-gold}{HTML}{E4A115} 

\newenvironment{PenList}{
  \begin{enumerate}[\textcolor{UNK-blue}{\PencilRightDown}]
    \addtolength{\itemsep}{-0.0\itemsep}}
  {\end{enumerate}}
  
\setbeamertemplate{theorems}[unnumbered]
\newtheorem{myprop}{Proposition}
\newtheorem{idea}{Idea}
\newtheorem{describ}{Description}
\newtheorem{Homework}{Homework}
\newtheorem{myproof}{Proof}
\newtheorem{fakeproof}{Fake Proof}
\newtheorem{require}{Requirement}
\newtheorem{myexample}{Example}
%------------------
\title[] % (optional, only for long titles)
{\textcolor{black}{\textbf{Natural Domain}} \\ 
\vspace{0.2in}
%\textcolor{black}{Mini-symposium} \\
%\textcolor{UNK-blue}{\textbf{University of Nebraska at Kearney}}
}

%\usepackage{amsmath}

\date{\today}


\makeatletter
\newcommand{\leqnomode}{\tagsleft@true\let\veqno\@@leqno}
\newcommand{\reqnomode}{\tagsleft@false\let\veqno\@@eqno}
\makeatother
\author[] % (optional, for multiple authors)
{Barton~Willis}
%\institute[UNK] % (optional)
%{
 % \inst{1}%
 %   Barton Willis \\
 %   Department of Mathematics and Statistics\\
 %   University Nebraska at Kearney  \\
 %   Kearney, Nebraska 68849  USA \\
 %   \phantom{xxxx}\\
  %  \href{mailto:willisb@unk.edu}{willisb@unk.edu}}
 



%\usepackage{hyperref}
\pgfplotsset{compat=1.18}
\begin{document}

\frame{\titlepage}

\begin{frame}{Natural domain}

Often the formula for a function is given, but not its domain. For
such functions, the standard is that the domain is the largest
set of real numbers such that the output of the function is a real number.

\begin{describ}[Natural Domain] The \emph{natural domain} of a function is the
        largest set on which the output of the function is a
        real number.  Examples of things that are \emph{not} real
        numbers include:
        \begin{itemize}
            \item division by zero
            \item square roots of negative numbers
            \item logarithms of non-positive numbers
        \end{itemize}
\end{describ}
    
\end{frame}
\begin{frame}

\textbf{Example} Find the natural domain of 
\begin{equation*}
    F(x) = \frac{1}{\frac{1}{x} + \frac{1}{x-2}}.
\end{equation*}

\begin{itemize}
    \item We need to require that \emph{every} denominator is nonzero.
    \item But that's tricky-there are three denominators; they are
    $x$, $x-2$, and $\frac{1}{x} + \frac{1}{x-2}$.
    
\end{itemize}

\textbf{Solution} In set builder notation, the natural domain is the set

\begin{equation*}
    \left\{ x \in \reals \bigg | \left(x \neq 0 \right)  
    \land \left(x-2 \neq 0 \right) \land  
    \left(\frac{1}{x} + \frac{1}{x-2} \neq 0 \right) \right\}
\end{equation*}

\begin{PenList}

    \item Arguably, this is a lovely answer. But if you were asked to express the
    set in interval notion, it's not such a lovely answer.

    \item To make it a truly lovely answer, we need to solve each inequation for 
    \item the variable (in this case  $x$).
    \item When we solve each inequation for the variable, we make the
    set builder notation \emph{explicit}.
    
\end{PenList}
\end{frame}

\begin{frame}{Explict Form}

Solving $ x \neq 0$ and $x-2 \neq 0$ is easy:
\[ 
    x \neq 0, \quad x \neq 2
\]
Solving \(\left(\frac{1}{x} + \frac{1}{x-2} \neq 0 \right) \) takes
a bit more work. Assuming that $x \neq 0$ and $x \neq 2$, we have
\begin{align*}
    \left[\frac{1}{x} + \frac{1}{x-2} \neq 0 \right] 
       &= \left[\frac{2 x-2}{{{x}^{2}}-2 x} \neq 0 \right],  &\mbox{(combine)}\\
       &= \left[2 x-2 \neq 0 \right], &(\mbox{algebra}) \\
       &=  \left[x  \neq 1 \right], &(\mbox{algebra})
\end{align*}
In \emph{explicit} set builder notation, the natural domain Is\
\begin{equation*}
    \left\{ x \in \reals \bigg | \left(x \neq 0\right)  
    \land \left(x  \neq 2 \right) \land  
    \left(x \neq 1 \right) \right\}
\end{equation*}
\end{frame}

\begin{frame}{Remove all Doubt}

If you are a bit uncertain of why the natural domain excludes $1$,
try evaluating at one:

\begin{equation*}
    \frac{1}{\frac{1}{1} + \frac{1}{1-2}} = \frac{1}{1 - 1} =
    \mbox{\textbf{Rubbish!}}
\end{equation*}

\vfill.
\end{frame}


\begin{frame}{Implicit requirement to be explicit}

    \begin{require} Unless stated otherwise, set builder notation
        must be explicit. The reason for this is that when expressed
        in explicit form, it's easy-peasy to express the set in either
        interval notion or pictorially.
    \end{require}
    
    \begin{myexample} Implicit (not allowed)
        \begin{equation*}
            \left\{ x \in \reals \bigg | \left(x \neq 0 \right)  
            \land \left(x-2 \neq 0 \right) \land  
            \left(\frac{1}{x} + \frac{1}{x-2} \neq 0 \right) \right\}.
        \end{equation*}
    Explicit (correct)
    \begin{equation*}
        \left\{ x \in \reals \bigg | \left(x \neq 0\right)  
        \land \left(x  \neq 2 \right) \land  
        \left(x \neq 1 \right) \right\}.
    \end{equation*}
    \end{myexample}
    
        
    \end{frame}
 
\begin{frame}{This is a Test}

\textbf{Question} The natural domain of 
\(\frac{1}{\frac{1}{x} + \frac{1}{x-2}} \) is 
\begin{equation*}
    \left\{ x \in \reals \bigg | \left(x \neq 0\right)  
    \land \left(x  \neq 2 \right) \land  
    \left(x \neq 1 \right) \right\}
\end{equation*}
What's the natural domain of the similar, but different expression   
\begin{equation*}
    107 + x^2 + \frac{46 + x}{\frac{1}{x} + \frac{1}{x-2}} 
\end{equation*}  

\textbf{Answer} Gathering the denominators, in implicit form, the
natural domain is
\begin{equation*}
    \left\{ x \in \reals \bigg | \left(x \neq 0 \right)  
    \land \left(x-2 \neq 0 \right) \land  
    \left(\frac{1}{x} + \frac{1}{x-2} \neq 0 \right) \right\}
\end{equation*}
This is no different from the condition for the domain of 
\(\frac{1}{\frac{1}{x} + \frac{1}{x-2}}\).

\end{frame}

\begin{frame}{Does it matter?}

\textbf{Question} So the \(107 + x^2\) and \(46 + x\) don't matter?

\textbf{Answer} The natural domain of \(\frac{1}{\frac{1}{x} + \frac{1}{x-2}} \)
and \( 107 + x^2 + \frac{46 + x}{\frac{1}{x} + \frac{1}{x-2}}  \) are
the same. 

This is the case because both expressions have identical denominators.

\vfill

\end{frame}

\begin{frame}{Resist Simplification}

\textbf{Question} Which calculation of the natural domain
of  \(\frac{1}{\frac{x-3}{x-4}} \) is correct?

\textbf{Answer 1}
\[
  \left  \{ x \big | \left(x-4 \neq 0\right) \land  
      \left(\frac{x-3}{x-4} \neq 0 \right) \right \} =
      \left \{ x \big | \left(x \neq 4 \right) \land  
      \left(x \neq 3 \right) \right \}.
\]

\textbf{Answer 2} First simplify \(\frac{1}{\frac{x-3}{x-4}} \) To
\(\frac{x-4}{x-3} \). Second the domain Is
\[
    \left \{ x \big | x-3 \neq 0  \right \} =
       \left \{ x \big | x \neq 3  \right \}.
 \]

\end{frame}


\end{document}
