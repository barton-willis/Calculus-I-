\documentclass[12pt,fleqn,answers]{exam}
\usepackage{pifont}
\usepackage{dingbat}
\usepackage{amsmath,amssymb}
\usepackage{epsfig}
\usepackage[super]{nth}
\usepackage[colorlinks=true,linkcolor=black,anchorcolor=black,citecolor=black,filecolor=black,menucolor=black,runcolor=black,urlcolor=black]{hyperref}
\usepackage[letterpaper, margin=0.75in]{geometry}
\addpoints
\boxedpoints
\pointsinmargin
\pointname{pts}

\usepackage[activate={true,nocompatibility},final,tracking=true,kerning=true,factor=1100,stretch=10,shrink=10]{microtype}
\usepackage[american]{babel}
%\usepackage[T1]{fontenc}
\usepackage{fourier}
\usepackage{isomath}
\usepackage{upgreek,amsmath}
\usepackage{amssymb}
\usepackage{graphicx}

\newcommand{\dotprod}{\, {\scriptzcriptztyle
    \stackrel{\bullet}{{}}}\,}

\newcommand{\reals}{\mathbf{R}}
\newcommand{\lub}{\mathrm{lub}} 
\newcommand{\glb}{\mathrm{glb}} 
\newcommand{\complex}{\mathbf{C}}
\newcommand{\dom}{\mbox{dom}}
\newcommand{\range}{\mbox{range}}
\newcommand{\cover}{{\mathcal C}}
\newcommand{\integers}{\mathbf{Z}}
\newcommand{\vi}{\, \mathbf{i}}
\newcommand{\vj}{\, \mathbf{j}}
\newcommand{\vk}{\, \mathbf{k}}
\newcommand{\bi}{\, \mathbf{i}}
\newcommand{\bj}{\, \mathbf{j}}
\newcommand{\bk}{\, \mathbf{k}}
\DeclareMathOperator{\Arg}{\mathrm{Arg}}
\DeclareMathOperator{\Ln}{\mathrm{Ln}}
\newcommand{\imag}{\, \mathrm{i}}

\usepackage{graphicx}
\usepackage{color}
\shadedsolutions
\definecolor{SolutionColor}{rgb}{0.8,0.9,1}
\newcommand\AM{\textsc{am}}
\newcommand\PM{\textsc{pm}}
     
\newcommand{\quiz}{2}
\newcommand{\term}{Fall}
\newcommand{\due}{Wednesday 31 August at 13:15 \PM}
\newcommand{\class}{MATH 115}
\begin{document}
\large


\vspace{0.1in}


\begin{questions} 

\question Find the solution set for the equation \( \frac{107}{1 + \exp(x)} = 0\).
\begin{solution}[1.0in]
Remember that $\left[\frac{a}{b}= 0\right] = [(a=0) \land (b \neq 0)]$. Using this fact,
we have
\begin{align*}
    \left[ \frac{107}{1 + \exp(x)} = 0 \right] &= \left[(107=0) \land (1 + \exp(x) \neq 0) \right] \\
             &= \varnothing
\end{align*}
\end{solution}

\question Inflation is causing the cost of chicken eggs to increase. The cost (in dollars) of a dozen chicken eggs is $C = 1.83 \times 1.052^T$, where $T$ is
the number of years after 1 January, 2022. When will chicken eggs cost \$2.00 per dozen?

\begin{solution}[1.0in]
\begin{align*}
    \left[2 = 1.83 \times 1.052^T\right] &= \left[\frac{2}{1.83}=1.052^T\right],\\
                                 &=\left[\ln\left(\frac{2}{1.83}\right)=T \ln(1.052)T\right],\\
                                 &=\left[T = \frac{\ln(\frac{2}{1.83})}{\ln(1.052)} \right],\\
                                 &= \left[T \approx 1.75 \mbox{ years}\right].
\end{align*}    
\end{solution}

\question Find the inverse of the function $W(x) = 5 x +1$ and $\dom(W) = \reals$.
\begin{solution}[3.0in]
\begin{align*}
    \left[ y = 5 x + 1, -\infty < x < \infty \right] &= \left[x = \frac{y-1}{5},  -\infty < \frac{y-1}{5} < \infty \right], \\
               &= \left[x = \frac{y-1}{5},  -\infty < y < \infty \right]
\end{align*}
So $W^{-1}(y) = \frac{y-1}{5}$ and $\dom(W^{-1}) = \reals$.

If $y$ is a real number, so is $\frac{y-1}{5}$. Thus the solution set of
\(-\infty < \frac{y-1}{5} < \infty \) is \\ \mbox{\(-\infty < y < \infty\).}

\end{solution}

\newpage
\question Define $Z(t) = t - 2 t^2 $. Find
\begin{parts}

\part \(\underset{[1, 1.1]}{\mbox{ARC}}(Z) = \)
\begin{solution}[1.0in]
\[
    \underset{[1, 1.1]}{\mbox{ARC}}(Z) = \frac{Z(1.1)-Z(1)}{1.1-1}=-3.2
\]
\end{solution}
\part \(\underset{[1, 1.01]}{\mbox{ARC}}(Z) = \)
\begin{solution}[1.0in]
    \[
        \underset{[1, 1.01]}{\mbox{ARC}}(Z) = \frac{Z(1.01)-Z(1)}{1.01-1}=-3.02
    \] 
\end{solution}

\part \(\underset{[1, 1.001]}{\mbox{ARC}}(Z) = \)
\begin{solution}[1.0in]
    \[
        \underset{[1, 1.001]}{\mbox{ARC}}(Z) = \frac{Z(1.001)-Z(1)}{1.001-1}=-3.002
    \] 
\end{solution}

\part a simplified formula for \(\underset{[1, b]}{\mbox{ARC}}(Z) = \)
\begin{solution}[1.0in]

We have $(-2 {{b}^{2}}+b+1) \div (b-1) = -2b-1$.  Using this
easy formula, parts `a' through `c' are easy.

\quad If you need a refresher on polynomial division, see 

\url{https://www.youtube.com/watch?v=RPXMBIFG_W4}

\end{solution}
\end{parts}
\end{questions}
\end{document}