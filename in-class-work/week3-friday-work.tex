\documentclass[12pt,answers, fleqn]{exam}
\usepackage{pifont}
\usepackage{dingbat}
\usepackage{amsmath,amssymb}
\usepackage{epsfig}
\usepackage[super]{nth}
\usepackage[colorlinks=true,linkcolor=black,anchorcolor=black,citecolor=black,filecolor=black,menucolor=black,runcolor=black,urlcolor=black]{hyperref}
\usepackage[letterpaper, margin=0.75in]{geometry}
\addpoints
\boxedpoints
\pointsinmargin
\pointname{pts}

\usepackage[activate={true,nocompatibility},final,tracking=true,kerning=true,factor=1100,stretch=10,shrink=10]{microtype}
\usepackage[american]{babel}
%\usepackage[T1]{fontenc}
\usepackage{fourier}
\usepackage{isomath}
\usepackage{upgreek,amsmath}
\usepackage{amssymb}
\usepackage{graphicx}

\newcommand{\dotprod}{\, {\scriptzcriptztyle
    \stackrel{\bullet}{{}}}\,}

\newcommand{\reals}{\mathbf{R}}
\newcommand{\lub}{\mathrm{lub}} 
\newcommand{\glb}{\mathrm{glb}} 
\newcommand{\complex}{\mathbf{C}}
\newcommand{\dom}{\mbox{dom}}
\newcommand{\range}{\mbox{range}}
\newcommand{\cover}{{\mathcal C}}
\newcommand{\integers}{\mathbf{Z}}
\newcommand{\vi}{\, \mathbf{i}}
\newcommand{\vj}{\, \mathbf{j}}
\newcommand{\vk}{\, \mathbf{k}}
\newcommand{\bi}{\, \mathbf{i}}
\newcommand{\bj}{\, \mathbf{j}}
\newcommand{\bk}{\, \mathbf{k}}
\DeclareMathOperator{\Arg}{\mathrm{Arg}}
\DeclareMathOperator{\Ln}{\mathrm{Ln}}
\newcommand{\imag}{\, \mathrm{i}}

\usepackage{graphicx}
\usepackage{color}
\shadedsolutions
\definecolor{SolutionColor}{rgb}{0.8,0.9,1}
\newcommand\AM{\textsc{am}}
\newcommand\PM{\textsc{pm}}
     
\newcommand{\quiz}{2}
\newcommand{\term}{Fall}
\newcommand{\due}{Wednesday 31 August at 13:15 \PM}
\newcommand{\class}{MATH 115}
\begin{document}
\large


\vspace{0.1in}
\noindent{\textbf{Week 3 Friday Work}}

\begin{questions} 

\question Find the value of each limit:

\begin{parts}

\part \(\displaystyle \lim_{x \to 1^{(-)}} \begin{cases} 3 & x < 1 \\ x & 1 \leq x \end{cases}\)
\begin{solution}[1.5in]
    We're looking at the limit from the left toward $1$. That allows us
    to simplify \(\begin{cases} 3 & x < 1 \\ x & 1 \leq x \end{cases} \) to $3$.
    Thus
    \begin{align*}
        \lim_{x \to 1^{(-)}} \begin{cases} 3 & x < 1 \\ x & 1 \leq x \end{cases} &=
        \lim_{x \to 1^{(-)}} 3   & \hfill \mbox{(simplification)} \\
        &= 3  &\hfill \mbox{(limit of constant)} \\
    \end{align*}
\end{solution}
\part \(\displaystyle \lim_{x \to 1^{(+)}} \begin{cases} 3 & x < 1 \\ x & 1 \leq x \end{cases}\)
\begin{solution}[1.5in]
    We're looking at the limit from the right toward $1$. That allows us
    to simplify \(\begin{cases} 3 & x < 1 \\ x & 1 \leq x \end{cases} \) to $x$.
    Thus
    \begin{align*}
        \lim_{x \to 1^{(-)}} \begin{cases} 3 & x < 1 \\ x & 1 \leq x \end{cases} &=
        \lim_{x \to 1^{(-)}} x   & \hfill \mbox{(simplification)} \\
        &= 1  &\hfill \mbox{(limit of constant)}) \\
    \end{align*}
    
\end{solution}
\part \(\displaystyle \lim_{x \to 1} \begin{cases} 3 & x < 1 \\ x & 1 \leq x \end{cases}\)
\begin{solution}[1.5in]
    From parts `a' and `b', we have  
    \(\displaystyle \lim_{x \to 1^{(-)}} \begin{cases} 3 & x < 1 \\ x & 1 \leq x \end{cases}
     \neq \lim_{x \to 1^{(+)}} \begin{cases} 3 & x < 1 \\ x & 1 \leq x \end{cases}\), so
     \(\displaystyle \lim_{x \to 1} \begin{cases} 3 & x < 1 \\ x & 1 \leq x \end{cases}\)
     does not exist (aka dne)
\end{solution}

\part \(\displaystyle \lim_{x \to 1} \begin{cases} 3 & x < 10 \\ \ln(x^x + 1)  \sin(1/x) & 10 \leq x \end{cases}\)
\begin{solution}[1.5in]
    The limit point is $1$. For $x$ near the limit point, we can
    simplify \(\displaystyle \lim_{x \to 1} \begin{cases} 3 & x < 10 \\ \ln(x^x + 1)  \sin(1/x) & 10 \leq x \end{cases}\) to $3$.
    Thus
    \[
        \lim_{x \to 1} \begin{cases} 3 & x < 10 \\ \ln(x^x + 1)  \sin(1/x) & 10 \leq x \end{cases}
         = \lim_{x \to 1} 3 = 3.
    \]
    
\end{solution}

\part \(\displaystyle \lim_{x \to 5} \frac{\sqrt{x + 2} - \sqrt{7}}{x - 5} \)
\begin{solution}[1.5in]
    Direct substitution is not an option. To start, let's do some tricky
    algebra:
    \begin{align*}
        \frac{\sqrt{x + 2} - \sqrt{7}}{x - 5} &= \frac{\sqrt{x + 2} - \sqrt{7}}{x - 5} \times \frac{\sqrt{x + 2} + \sqrt{7}}{\sqrt{x + 2} + \sqrt{7}}, \\
           &= \frac{x+2 - 7}{(x-5)(\sqrt{x + 2} + \sqrt{7})}, \\
           &= \frac{1}{\sqrt{x + 2} + \sqrt{7}}.
    \end{align*}

\end{solution}


\part \(\displaystyle \lim_{x \to \uppi} \frac{\sqrt{x + \uppi} - \sqrt{2 \uppi}}{x - \uppi} \)
\begin{solution}[1.5in]
\end{solution}

\part \(\displaystyle \lim_{x \to 3} \frac{\sqrt{x + \uppi} - \sqrt{2 \uppi}}{x - \uppi} \)
\begin{solution}[1.5in]
\end{solution}

\part \(\displaystyle \lim_{x \to \sqrt{107}} \frac{x}{|x|}  \)
\begin{solution}[1.5in]
\end{solution}

\part \(\displaystyle \lim_{x \to -\sqrt{107}} \frac{x}{|x|}  \)
\begin{solution}[0.0in]
\end{solution}
\end{parts}
\end{questions}
\end{document}