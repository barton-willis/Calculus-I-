\documentclass[12pt,fleqn,answers]{exam}
\usepackage{pifont}
\usepackage{dingbat}
\usepackage{amsmath,amssymb}
\usepackage{epsfig}
\usepackage{upgreek}
\usepackage[super]{nth}
\usepackage[colorlinks=true,linkcolor=black,anchorcolor=black,citecolor=black,filecolor=black,menucolor=black,runcolor=black,urlcolor=black]{hyperref}
\usepackage[letterpaper, margin=0.75in]{geometry}
\addpoints
\boxedpoints
\pointsinmargin
\pointname{pts}

\usepackage[activate={true,nocompatibility},final,tracking=true,kerning=true,factor=1100,stretch=10,shrink=10]{microtype}
\usepackage[american]{babel}
%\usepackage[T1]{fontenc}
\usepackage{fourier}
\usepackage{isomath}
\usepackage{upgreek,amsmath}
\usepackage{amssymb}
\usepackage{graphicx}

\newcommand{\dotprod}{\, {\scriptzcriptztyle
    \stackrel{\bullet}{{}}}\,}

\newcommand{\reals}{\mathbf{R}}
\newcommand{\lub}{\mathrm{lub}} 
\newcommand{\glb}{\mathrm{glb}} 
\newcommand{\complex}{\mathbf{C}}
\newcommand{\dom}{\mbox{dom}}
\newcommand{\range}{\mbox{range}}
\newcommand{\cover}{{\mathcal C}}
\newcommand{\integers}{\mathbf{Z}}
\newcommand{\vi}{\, \mathbf{i}}
\newcommand{\vj}{\, \mathbf{j}}
\newcommand{\vk}{\, \mathbf{k}}
\newcommand{\bi}{\, \mathbf{i}}
\newcommand{\bj}{\, \mathbf{j}}
\newcommand{\bk}{\, \mathbf{k}}
\DeclareMathOperator{\Arg}{\mathrm{Arg}}
\DeclareMathOperator{\Ln}{\mathrm{Ln}}
\newcommand{\imag}{\, \mathrm{i}}

\usepackage{graphicx}
\usepackage{color}
\shadedsolutions
\definecolor{SolutionColor}{rgb}{0.8,0.9,1}
\newcommand\AM{\textsc{am}}
\newcommand\PM{\textsc{pm}}
     
\newcommand{\quiz}{4}
\newcommand{\term}{Fall}
\newcommand{\due}{Wednesday 14 September 13:15 \PM}
\newcommand{\class}{MATH 115}
\begin{document}
\large




\noindent we have 
\begin{description}

\item[Rule \#0 (polynomial)] Every polynomial is continuous everywhere.

\item[Rule \#1 (rational)] Every rational function is continuous everywhere it's defined.

\item[Rule \#2]  Each of the following functions are continuous everywhere they
are defined: power (both integer and noninteger powers), trigonometric, inverse trigonometric, exponential, and logarithmic.

\item[Rule \#3] Let $F$ and $G$ be functions that are continuous at $c$ and let
$a,b$ be numbers and let $n$ be a positive integer; each of the following are 
continuous at $c$:
\begin{align*}
    &a F + b G \\
    &F G & \\
    &F/G & (\mbox{ provided  $G(c) \neq 0$) \\
    &F^n & \\
    &F^{1/n} & (\mbox{ provided $F{1/n}$ is defined on a neighborhood of $c$})
\end{align*}

\item[Rule \#4] Let $G$ be continuous at $c$ and let $F$ be continuous at $F(c)$.
Then $F \circ G$ is continuous at $c$.

\end{description}
\end{document}
