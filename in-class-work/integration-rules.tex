\documentclass[10pt,fleqn,answers]{exam}
\usepackage{pifont}
\usepackage{dingbat}
\usepackage{amsmath,amssymb}
\usepackage{epsfig}
\usepackage{upgreek}
\usepackage[super]{nth}
\usepackage[colorlinks=true,linkcolor=black,anchorcolor=black,citecolor=black,filecolor=black,menucolor=black,runcolor=black,urlcolor=black]{hyperref}
\usepackage[letterpaper, margin=0.5in]{geometry}
\addpoints
\boxedpoints
\pointsinmargin
\pointname{pts}

\usepackage[activate={true,nocompatibility},final,tracking=true,kerning=true,factor=1100,stretch=10,shrink=10]{microtype}
\usepackage[american]{babel}
%\usepackage[T1]{fontenc}
\usepackage{fourier}
\usepackage{isomath}
\usepackage{upgreek,amsmath}
\usepackage{amssymb}
\usepackage{graphicx}

\newcommand{\dotprod}{\, {\scriptzcriptztyle
    \stackrel{\bullet}{{}}}\,}

\newcommand{\reals}{\mathbf{R}}
\newcommand{\lub}{\mathrm{lub}} 
\newcommand{\glb}{\mathrm{glb}} 
\newcommand{\complex}{\mathbf{C}}
\newcommand{\dom}{\mbox{dom}}
\newcommand{\range}{\mbox{range}}
\newcommand{\cover}{{\mathcal C}}
\newcommand{\integers}{\mathbf{Z}}
\newcommand{\vi}{\, \mathbf{i}}
\newcommand{\vj}{\, \mathbf{j}}
\newcommand{\vk}{\, \mathbf{k}}
\newcommand{\bi}{\, \mathbf{i}}
\newcommand{\bj}{\, \mathbf{j}}
\newcommand{\bk}{\, \mathbf{k}}
\DeclareMathOperator{\Arg}{\mathrm{Arg}}
\DeclareMathOperator{\Ln}{\mathrm{Ln}}
\newcommand{\imag}{\, \mathrm{i}}

\usepackage{graphicx}
\usepackage{color}
\shadedsolutions
\definecolor{SolutionColor}{rgb}{0.8,0.9,1}
\newcommand\AM{\textsc{am}}
\newcommand\PM{\textsc{pm}}
     
\newcommand{\quiz}{4}
\newcommand{\term}{Fall}
\newcommand{\due}{Wednesday 14 September 13:15 \PM}
\newcommand{\class}{MATH 115}
\begin{document}
\large




\begin{description}

\item[Rule \#0 (power rule)]  For $n \in \reals_{\neq -1}$, we have  
\begin{equation*}
 \int x^n \,  \mathrm{d} x = \frac{1}{n+1} x^{n+1}+C
 \end{equation*}
  is valid on any interval on which $x \mapsto x^n$ is defined. 

\item[Rule \#1 (reciprocal rule)]  On any interval that does not contain zero, we have 
\begin{equation*}
 \int \frac{1}{x} \,   \mathrm{d} x = \ln(|x|) + C
 \end{equation*}

\item[Rule \#2 (outative)]  For all $a \in \reals$ and for any function $F$ that have an antiderivative on an interval $I$, 
we have 
\begin{equation*}
  \int a F(x) \, \mathrm{d} x =  a \int F(x) \, \mathrm{d} x
 \end{equation*}
 on the interval $I$.

\item[Rule \#3 (additive)]  For all $a \in \reals$ and for two functions  $F$  and $G$ that have an antiderivative  on an interval $I$,  we have 
\begin{equation*}
  \int  F(x)  + G(x) \, \mathrm{d} x =   \int F(x) \, \mathrm{d} x + \int G(x) \, \mathrm{d} x 
\end{equation*}
 on the interval $I$.

\item[Rule \#4 (linear)] For all $a,b \in \reals$ and for two functions  $F$  and $G$ that have an antiderivative  on an interval $I$,  we have 
\begin{equation*}
  \int  a F(x)  + b G(x) \, \mathrm{d} x =   a \int F(x) \, \mathrm{d} x + b \int G(x) \, \mathrm{d} x 
\end{equation*}
This rule is a combination of the outative rule (Rule 2) and the additive rule (Rule 3).
 on the interval $I$.

\item[Rule \#5 (special cases)] For all $a \in \reals_{\neq 0}$, for the interval $\reals$, we have
\begin{align*}
 &\int \cos(a x) \, \mathrm{d} x = \frac{1}{a} \sin(a x) + C,\\
 &\int \sin(a x) \, \mathrm{d} x = -\frac{1}{a} \cos(a x) + C,\\
 &\int \mathrm{e}^{a x} \, \mathrm{d} x = \frac{1}{a} \mathrm{e}^{a x} + C.
\end{align*}


\end{description}
\end{document}
