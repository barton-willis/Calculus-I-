\documentclass[12pt,fleqn,answers]{exam}
\usepackage{pifont}
\usepackage{dingbat}
\usepackage{amsmath,amssymb}
\usepackage{epsfig}
\usepackage{upgreek}
\usepackage[super]{nth}
\usepackage[colorlinks=true,linkcolor=black,anchorcolor=black,citecolor=black,filecolor=black,menucolor=black,runcolor=black,urlcolor=black]{hyperref}
\usepackage[letterpaper, margin=0.75in]{geometry}
\addpoints
\boxedpoints
\pointsinmargin
\pointname{pts}

\usepackage[activate={true,nocompatibility},final,tracking=true,kerning=true,factor=1100,stretch=10,shrink=10]{microtype}
\usepackage[american]{babel}
%\usepackage[T1]{fontenc}
\usepackage{fourier}
\usepackage{isomath}
\usepackage{upgreek,amsmath}
\usepackage{amssymb}
\usepackage{graphicx}

\newcommand{\dotprod}{\, {\scriptzcriptztyle
    \stackrel{\bullet}{{}}}\,}

\newcommand{\reals}{\mathbf{R}}
\newcommand{\lub}{\mathrm{lub}} 
\newcommand{\glb}{\mathrm{glb}} 
\newcommand{\complex}{\mathbf{C}}
\newcommand{\dom}{\mbox{dom}}
\newcommand{\range}{\mbox{range}}
\newcommand{\cover}{{\mathcal C}}
\newcommand{\integers}{\mathbf{Z}}
\newcommand{\vi}{\, \mathbf{i}}
\newcommand{\vj}{\, \mathbf{j}}
\newcommand{\vk}{\, \mathbf{k}}
\newcommand{\bi}{\, \mathbf{i}}
\newcommand{\bj}{\, \mathbf{j}}
\newcommand{\bk}{\, \mathbf{k}}
\DeclareMathOperator{\Arg}{\mathrm{Arg}}
\DeclareMathOperator{\Ln}{\mathrm{Ln}}
\newcommand{\imag}{\, \mathrm{i}}

\usepackage{graphicx}
\usepackage{color}
\shadedsolutions
\definecolor{SolutionColor}{rgb}{0.8,0.9,1}
\newcommand\AM{\textsc{am}}
\newcommand\PM{\textsc{pm}}
     
\newcommand{\quiz}{4}
\newcommand{\term}{Fall}
\newcommand{\due}{Wednesday 14 September 13:15 \PM}
\newcommand{\class}{MATH 115}
\begin{document}
\large




\noindent Let $n \in \reals$ and let $a \in \reals$. Let $x$ be the independent variable and let $u$ and $v$ be dependent variables. Where every both $u$ and $v$are differentiable, we have
\begin{description}

\item[Rule \#0 (constant rule)]  \( \frac{\mathrm{d}}{\mathrm{d} x} \left [a \right] = 0 \)
\item[Rule \#1 (power rule)]  \( \frac{\mathrm{d}}{\mathrm{d} x} \left [x^n \right] = n x^{n-1}\).  When $n=1$, the rule
is \( \frac{\mathrm{d}}{\mathrm{d} x} \left [x \right] = 1 \).
\item[Rule \#2 (outative rule)]  \( \frac{\mathrm{d}}{\mathrm{d} x} \left [a u \right] = a \frac{\mathrm{d} u}{\mathrm{d} x}\).
\item[Rule \#3 (additive rule)]  \( \frac{\mathrm{d}}{\mathrm{d} x} \left [ u + v \right] =  \frac{\mathrm{d} u}{\mathrm{d} x}
+   \frac{\mathrm{d} v}{\mathrm{d} x}\).
\item[Rule \#4 (product  rule)]  \( \frac{\mathrm{d}}{\mathrm{d} x} \left [ u v \right] = \frac{\mathrm{d} u}{\mathrm{d} x} v
+ u \frac{\mathrm{d} v}{\mathrm{d} x}\)
\item[Rule \#5 (quotient  rule)]  \( \frac{\mathrm{d}}{\mathrm{d} x} \left [\frac{u}{v} \right] = 
\frac{v \frac{\mathrm{d} u}{\mathrm{d} x} - u \frac{\mathrm{d} v}{\mathrm{d} x}}{v^2}\)

\item[Rule \#6 (exponential  rule)] \( \frac{\mathrm{d}}{\mathrm{d} x} \left [\mathrm{e}^x \right] = \mathrm{e}^x \)

 \item[Rule \#7 (absolute value rule)] \( \frac{\mathrm{d}}{\mathrm{d} x} \left[|x| \right] = 
 \begin{cases} -1 & x < 0 \\   1 & x > 0 \end{cases}\). The absolute value function is \emph{not} differentiable at zero.


\end{description}

\noindent Notice: The product and quotient rules require  that both factors are differentiable. When either fails to be differentiable, anything can happen.

\end{document}
