\documentclass[12pt,fleqn,answers]{exam}
\usepackage{pifont}
\usepackage{dingbat}
\usepackage{amsmath,amssymb}
\usepackage{epsfig}
\usepackage{upgreek}
\usepackage[super]{nth}
\usepackage[colorlinks=true,linkcolor=black,anchorcolor=black,citecolor=black,filecolor=black,menucolor=black,runcolor=black,urlcolor=black]{hyperref}
\usepackage[letterpaper, margin=0.75in]{geometry}
\addpoints
\boxedpoints
\pointsinmargin
\pointname{pts}

\usepackage[activate={true,nocompatibility},final,tracking=true,kerning=true,factor=1100,stretch=10,shrink=10]{microtype}
\usepackage[american]{babel}
%\usepackage[T1]{fontenc}
\usepackage{utopia}
\usepackage{isomath}
\usepackage{upgreek,amsmath}
\usepackage{amssymb}
\usepackage{graphicx}

\newcommand{\dotprod}{\, {\scriptzcriptztyle\stackrel{\bullet}{{}}}\,}

\newcommand{\reals}{\mathbf{R}}
\newcommand{\lub}{\mathrm{lub}} 
\newcommand{\glb}{\mathrm{glb}} 
\newcommand{\complex}{\mathbf{C}}
\newcommand{\dom}{\mbox{dom}}
\newcommand{\range}{\mbox{range}}
\newcommand{\cover}{{\mathcal C}}
\newcommand{\integers}{\mathbf{Z}}
\newcommand{\vi}{\, \mathbf{i}}
\newcommand{\vj}{\, \mathbf{j}}
\newcommand{\vk}{\, \mathbf{k}}
\newcommand{\bi}{\, \mathbf{i}}
\newcommand{\bj}{\, \mathbf{j}}
\newcommand{\bk}{\, \mathbf{k}}
\DeclareMathOperator{\Arg}{\mathrm{Arg}}
\DeclareMathOperator{\Ln}{\mathrm{Ln}}
\newcommand{\imag}{\, \mathrm{i}}

\usepackage{graphicx}
\usepackage{color}
\shadedsolutions
\definecolor{SolutionColor}{rgb}{0.8,0.9,1}
\newcommand\AM{\textsc{am}}
\newcommand\PM{\textsc{pm}}
     
\newcommand{\quiz}{7}
\newcommand{\term}{Fall}
\newcommand{\due}{Wednesday 5 October at 13:15 \PM}
\newcommand{\class}{MATH 115}
\begin{document}
\large
\vspace{0.1in}
\noindent\makebox[3.0truein][l]{\textbf{\class}}
\textbf{Name:} \hrulefill \\
\noindent \makebox[3.0truein][l]{\textbf{In class work \quiz, \term \/ \the\year}}
\textbf{Row and Seat}:\hrulefill\\
\vspace{0.1in}


\noindent  In class work  \quiz\/  has questions 1 through  \numquestions \/ with a total of  \numpoints\/  points.   
Turn in your work at the end of class  \emph{on paper}. This assignment is due \emph{\due}.

\vspace{0.1in}


\begin{questions} 

    \question Find $\displaystyle \frac{\mathrm{d} y}{\mathrm{d} x}$
    and $\displaystyle \frac{\mathrm{d}^2 y}{\mathrm{d} x^2}$ evaluated
    at $(x= 1/\sqrt{2},y=1/\sqrt{2})$ given $x^2+ y^2 = 1$.
    \begin{solution}[3.5in]
        \begin{equation*}
            \frac{\mathrm{d}}{\mathrm{d} x} \left [x^2 + y^2 = 1 \right] 
                = \left[2x + 2 y \frac{\mathrm{d} y}{\mathrm{d} x} = 0 \right].
        \end{equation*}
Solving this for $\frac{\mathrm{d} y}{\mathrm{d} x}$ gives
\begin{equation*}
    \frac{\mathrm{d} y}{\mathrm{d} x} = - \frac{x}{y}.
\end{equation*}
    Pasting in the data $x \leftarrow 1/\sqrt{2}, y \leftarrow 1/\sqrt{2}$ gives  
    $ \left. \frac{\mathrm{d} y}{\mathrm{d} x}  \right \vert_{ x =1/\sqrt{2} y = 1/\sqrt{2}} = - 1$.
 
 To find the second derivative, we differentiate \( \frac{\mathrm{d} y}{\mathrm{d} x} = - \frac{x}{y} \).  This gives
 \[
    \frac{\mathrm{d}^2 y}{\mathrm{d} x^2} =    \frac{\mathrm{d}}{\mathrm{d} x} \left [ - \frac{x}{y} \right]
     = -\frac{y - x y^\prime}{y^2}.
 \]
       Pasting  in the data $x \leftarrow 1/\sqrt{2}, y \leftarrow 1/\sqrt{2}$ gives  
 \[
    \left . \frac{\mathrm{d}^2 y}{\mathrm{d} x^2} \right \vert_{ x =1/\sqrt{2} y = 1/\sqrt{2}} =   -\frac{1/\sqrt{2}  + 1/\sqrt{2} }{1/2} = -2 \sqrt{2}.
 \]       
    \end{solution}

    \question [1] The equation $x y = y - 1 + \mathrm{e}^{-y} $ defines\footnote{This problem is motivated by an unpublished mathematical model of hemoglobin glycation.}
    $y$ as a function of $x$.  Find a formula for $\displaystyle \frac{\mathrm{d} y}{\mathrm{d} x}$.
    
    \begin{solution}[2.5in]
    We have
    \begin{align*}
      \frac{\mathrm{d}}{\mathrm{d} x} \left [ x y = y - 1 + \mathrm{e}^{-y} \right] &= \left[ y + x y^\prime = y^\prime - y^\prime \mathrm{e}^{-y} \right], \\
      &= \left[ y  =  (1-x  - \mathrm{e}^{-y}) y^\prime  \right], \\
      &= \left[ y^\prime = \frac{y}{1-x - \mathrm{e}^{-y}} \right].
    \end{align*}
    \end{solution}

    %\newpage 

    \question Find a formula for each derivative
    \begin{parts}
 
    \part [1] \(\displaystyle \frac{\mathrm{d}}{\mathrm{d} x} \left [ \ln(x (x-1)) \right ] \)
    \begin{solution}[1.5in]
      \[
        \frac{\mathrm{d} y}{\mathrm{d} x} = \frac{2 x-1}{\left( x-1\right)  x} = \frac{1}{x}+\frac{1}{x-1}
      \]
    \end{solution}

    \part[1] \(\displaystyle \frac{\mathrm{d}}{\mathrm{d} x} \left [ \tan^{-1}(x^2)) \right ] \)
    \begin{solution}[1.5in]
    \[
        \frac{\mathrm{d} y}{\mathrm{d} x} = \frac{2 x}{{{x}^{4}}+1}.
    \]
    \end{solution}

    \part[1] \(\displaystyle \frac{\mathrm{d}}{\mathrm{d} x} \left [ \csc^{-1}(1/x^2)) \right ] \)
    \begin{solution}[1.5in]
    \[
         \frac{\mathrm{d} y}{\mathrm{d} x} =     \frac{2 x}{\sqrt{1-{{x}^{4}}}}
    \] 
    \end{solution}


    \part[1] \(\displaystyle \frac{\mathrm{d}}{\mathrm{d} x} \left [ x \tan^{-1}(x) \right ] \)
    \begin{solution}[1.5in]
    \[
         \frac{\mathrm{d} y}{\mathrm{d} x} =    \tan^{-1} (x)+\frac{x}{{{x}^{2}}+1}
    \] 
    \end{solution}

    \part [1] \(\displaystyle \frac{\mathrm{d}}{\mathrm{d} x} \left [ \cot^{-1}(x) + \tan^{-1}(x) \right ] \)
    \begin{solution}%[1.5in]
      \[
        \frac{\mathrm{d} y}{\mathrm{d} x} = -\frac{1}{1+x^2} + \frac{1}{1+x^2} = 0.
        \]
    \end{solution}
    \end{parts}

\end{questions}


\end{document}

