\documentclass[12pt,fleqn,answers]{exam}
\usepackage{pifont}
\usepackage{dingbat}
\usepackage{amsmath,amssymb}
\usepackage{epsfig}
\usepackage[super]{nth}
\usepackage[colorlinks=true,linkcolor=black,anchorcolor=black,citecolor=black,filecolor=black,menucolor=black,runcolor=black,urlcolor=black]{hyperref}
\usepackage[letterpaper, margin=0.5in]{geometry}
\addpoints
\boxedpoints
\pointsinmargin
\pointname{pts}

\usepackage[activate={true,nocompatibility},final,tracking=true,kerning=true,factor=1100,stretch=10,shrink=10]{microtype}
\usepackage[american]{babel}
%\usepackage[T1]{fontenc}
\usepackage{fourier}
\usepackage{isomath}
\usepackage{upgreek,amsmath}
\usepackage{amssymb}
\usepackage{graphicx}

\newcommand{\dotprod}{\, {\scriptzcriptztyle
    \stackrel{\bullet}{{}}}\,}

\newcommand{\reals}{\mathbf{R}}
\newcommand{\lub}{\mathrm{lub}} 
\newcommand{\glb}{\mathrm{glb}} 
\newcommand{\complex}{\mathbf{C}}
\newcommand{\dom}{\mbox{dom}}
\newcommand{\range}{\mbox{range}}
\newcommand{\cover}{{\mathcal C}}
\newcommand{\integers}{\mathbf{Z}}
\newcommand{\vi}{\, \mathbf{i}}
\newcommand{\vj}{\, \mathbf{j}}
\newcommand{\vk}{\, \mathbf{k}}
\newcommand{\bi}{\, \mathbf{i}}
\newcommand{\bj}{\, \mathbf{j}}
\newcommand{\bk}{\, \mathbf{k}}
\DeclareMathOperator{\Arg}{\mathrm{Arg}}
\DeclareMathOperator{\Ln}{\mathrm{Ln}}
\newcommand{\imag}{\, \mathrm{i}}

\usepackage{graphicx}
\usepackage{color}
\shadedsolutions
\definecolor{SolutionColor}{rgb}{0.8,0.9,1}
\newcommand\AM{\textsc{am}}
\newcommand\PM{\textsc{pm}}
     
\usepackage{utopia}

\newcommand{\quiz}{8}
\newcommand{\term}{Fall}
\newcommand{\due}{Wednesday 31 August at 13:15 \PM}
\newcommand{\class}{MATH 115}
\begin{document}



\vspace{0.1in}

\noindent \textbf{Week 8 FLO} (Friday Learning Opportunity) 

\begin{questions}  
    
  \question Find a TL to $y = \sqrt[3]{x}$ with a point of tangency of
  $(x=1000, y = 10)$. 
  
  \begin{solution}%[3.5in]
  We have $\displaystyle \frac{\mathrm{d} y}{\mathrm{d} x} = \frac{1}{3} x^{-2/3}$. So
  $ \left. \displaystyle \frac{\mathrm{d} y}{\mathrm{d} x}  \right \vert_{x=1000} = \frac{1}{300}$. So 
\[
   L(x) = 10 + \frac{1}{300} (x-1000).
\]

    
  \end{solution}

  \question Use the TL from Question 1 to estimate $\sqrt[3]{1007}$. Compare this to is 
  true value.

  \begin{solution}%[3.5in]
\[
    \sqrt[3]{1007} \approx L(1007) = 10 + \frac{1}{300} (1007-1000) = \frac{3007}{300} \approx 10.023.
\]
The true value is approximately $10.0232790996342627192912842164323354433109773879416081626$. So 10.023
is a pretty good estimate.
    
  \end{solution}

 
  %\newpage

  \question Assume an elliptical pancake.  As a cornmeal pancake expands on the skillet,
  its semi-major axis remains three times its semi-minor axis.  At the moment the surface 
  area of the pancake is 12 square inches, the rate of change of its area is $1/2$ 
  square inches per second.  At this moment, find the rate of change of its semi-minor
  axis.

  \begin{solution}[3.5in] Let the semi-major axis be $a$ and the semi-minor axis be $b$. We are
  given that $a = 3 b$.  The area $A$ is $A = \uppi a b = 3 \uppi b^2$. So
  \begin{equation}
    \frac{\mathrm{d} A}{\mathrm{d} t} = 6 \uppi b  \frac{\mathrm{d} b}{\mathrm{d} t}.
  \end{equation}
  Pasting in the data gives
   \begin{equation}
   12 = 3 \uppi b^2   \quad  \frac{1}{2}  = 6 \uppi b  \frac{\mathrm{d} b}{\mathrm{d} t}.
  \end{equation}
  Solving for $b$ and $\frac{\mathrm{d} b}{\mathrm{d} t}$ gives $ \frac{\mathrm{d} b}{\mathrm{d} t} =
  \frac{1}{24 \sqrt{\ensuremath{\pi} }}$.
  \end{solution}


  \question A monarch butterfly is flying directly south and a goldfinch is flying directly 
  east.  As some moment, the monarch is 3 miles south of the UNK bell tower and the goldfinch
  is five miles east. Also at this moment, the speed of the monarch is 8 mph and the 
  speed of the goldfinch is 47 mph.  At this moment, what is the rate of change of
  the distance between the monarch and the goldfinch?

\begin{solution}
  Let the position of the monarch be $y$, the position of 
  the goldfinch be $x$, and distance in between them be $R$.
  %A relation between these variables is $R^2 = x^2 + y^$.
  Differentiating with respect to $t$ gives
  \begin{equation}
    2 R \frac{\mathrm{d} R}{\mathrm{d} t} =
    2 x \frac{\mathrm{d} x}{\mathrm{d} t}
    + 2 y \frac{\mathrm{d} y}{\mathrm{d} t}.
      \end{equation}
Pasting in the data gives the equations
\begin{align*}
     R^2 &= 34, \\
     2 R \frac{\mathrm{d} R}{\mathrm{d} t} &=
     2 \times 5 \times 47
     + 2 \times 3  \times 8.
\end{align*}
So
\[
  \frac{\mathrm{d} R}{\mathrm{d} t} =  \frac{259}{\sqrt{34}} \,\, \mbox{mph}
\]
\end{solution}
\end{questions}
\end{document}