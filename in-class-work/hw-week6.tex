\documentclass[12pt,fleqn]{exam}
\usepackage{pifont}
\usepackage{dingbat}
\usepackage{amsmath,amssymb}
\usepackage{epsfig}
\usepackage{upgreek}
\usepackage[super]{nth}
\usepackage[colorlinks=true,linkcolor=black,anchorcolor=black,citecolor=black,filecolor=black,menucolor=black,runcolor=black,urlcolor=black]{hyperref}
\usepackage[letterpaper, margin=0.75in]{geometry}
\addpoints
\boxedpoints
\pointsinmargin
\pointname{pts}

\usepackage[activate={true,nocompatibility},final,tracking=true,kerning=true,factor=1100,stretch=10,shrink=10]{microtype}
\usepackage[american]{babel}
%\usepackage[T1]{fontenc}
\usepackage{fourier}
\usepackage{isomath}
\usepackage{upgreek,amsmath}
\usepackage{amssymb}
\usepackage{graphicx}

\newcommand{\dotprod}{\, {\scriptzcriptztyle\stackrel{\bullet}{{}}}\,}

\newcommand{\reals}{\mathbf{R}}
\newcommand{\lub}{\mathrm{lub}} 
\newcommand{\glb}{\mathrm{glb}} 
\newcommand{\complex}{\mathbf{C}}
\newcommand{\dom}{\mbox{dom}}
\newcommand{\range}{\mbox{range}}
\newcommand{\cover}{{\mathcal C}}
\newcommand{\integers}{\mathbf{Z}}
\newcommand{\vi}{\, \mathbf{i}}
\newcommand{\vj}{\, \mathbf{j}}
\newcommand{\vk}{\, \mathbf{k}}
\newcommand{\bi}{\, \mathbf{i}}
\newcommand{\bj}{\, \mathbf{j}}
\newcommand{\bk}{\, \mathbf{k}}
\DeclareMathOperator{\Arg}{\mathrm{Arg}}
\DeclareMathOperator{\Ln}{\mathrm{Ln}}
\newcommand{\imag}{\, \mathrm{i}}

\usepackage{graphicx}
\usepackage{color}
\shadedsolutions
\definecolor{SolutionColor}{rgb}{0.8,0.9,1}
\newcommand\AM{\textsc{am}}
\newcommand\PM{\textsc{pm}}
     
\newcommand{\quiz}{6}
\newcommand{\term}{Fall}
\newcommand{\due}{Wednesday 27 September 13:15 \PM}
\newcommand{\class}{MATH 115}
\begin{document}
\large
\vspace{0.1in}
\noindent\makebox[3.0truein][l]{\textbf{\class}}
\textbf{Name:} \hrulefill \\
\noindent \makebox[3.0truein][l]{\textbf{In class work \quiz, \term \/ \the\year}}
\textbf{Row and Seat}:\hrulefill\\
\vspace{0.1in}


\noindent  In class work  \quiz\/  has questions 1 through  \numquestions \/ with a total of  \numpoints\/  points.   
Turn in your work at the end of class  \emph{on paper}. This assignment is due \emph{\due}.

\vspace{0.1in}


\begin{questions} 

    \question As a function of time $t$ (seconds), the position $s$  (feet) 
    of a 44 horsepower 1952 \mbox{Farmall \textregistered} \,  tractor moving along a flat piece of 
    Floyd Creek Road 
    is given by \mbox{$s = 2 t^{3/2}$}. This relation holds
    for $1 \leq t \leq 9$. Since our friends in the Physics department
    might be watching, we should append the correct units to all answers:
    for example, \(\frac{\mathrm{d} s}{\mathrm{d} t} = 46 \mbox{ ft}/\mbox{sec} \)
    and \(\frac{\mathrm{d}^2 s}{\mathrm{d} t^2} = 107 \mbox{ ft}/\mbox{sec}^2 \),
    and \textbf{not} \(\frac{\mathrm{d} s}{\mathrm{d} t} = 46 \)
    and \(\frac{\mathrm{d}^2 s}{\mathrm{d} t^2} = 107 \).

    \begin{parts}

    \part [1] Find the \emph{displacement} of the tractor on the time
    interval $[1,9]$. 
    \begin{solution}[1.5in]   
        
    \end{solution}

    \part [1] Find the \emph{average velocity} of the tractor on the time
    interval $[1,9]$. 
    \begin{solution}[1.5in]   
        
    \end{solution}

    \part [1] Find the \emph{velocity} of the tractor at the time $t=1$.
    That is, find 
    $\displaystyle \left . \frac{\mathrm{d} s}{\mathrm{d} t} 
    \right \vert_{t=1}$.
    \begin{solution}[1.5in]   
        
    \end{solution}

    \part [1] Find the \emph{velocity} of the tractor at the time $t=9$.
    That is, find 
    $\displaystyle \left . \frac{\mathrm{d} s}{\mathrm{d} t} 
    \right \vert_{t=9}$.
    \begin{solution}%[2.5in]   
        
    \end{solution}

    \newpage 
    \part [1] Find the \emph{acceleration} of the tractor at the time $t=2$.
    That is, find 
    $\displaystyle \left . \frac{\mathrm{d}^2 s}{\mathrm{d} t^2} 
    \right \vert_{t=2}$.
    \begin{solution}[2.5in]   
        
    \end{solution}

 

    \part [1] Show that $a v$, where $v$ is the velocity and $a$ is
    the acceleration of the tractor is a constant\footnote{For acceleration with a 
    constant power, $a v$ is constant. For the most part, internal 
    combustion engines deliver constant power at any speed, making acceleration of
    an automobile different from acceleration with a constant force.
    But starting from a stop, initially a car accelerates with a 
    constant force before it transitions to acceleration with a
    constant power. Physics texts are full of problems involving
    acceleration with a constant force, but problems involving
    acceleration with a power are obscure.}    for times $t$ 
    in the interval $[1,9]$.
    \begin{solution}[2.5in]   
        
    \end{solution}
    
    \end{parts}

    \newpage
    \question The position $s$ of my pet American Fuzzy Lop rabbit Wilber moving along a line as a function of 
    time $t$ 
    is $s = \frac{1}{2} t^2 - 2t + 4$, where we consider positive values of $s$ to be
    to the right and negative to the left.

    \begin{parts}

         \part[1] When is Wilber moving to the right? That is, when is 
         $\displaystyle \frac{\mathrm{d} s}{\mathrm{d} t} > 0$?

         \begin{solution}[2.5in]   
        
         \end{solution}
         
         \part[1] When is Wilber moving to the left? That is, when is 
         $\displaystyle \frac{\mathrm{d} s}{\mathrm{d} t} < 0$?

         \begin{solution}[2.5in]   
        
         \end{solution}

         \part[1] Find Wilber's \emph{speed} when $t = 2$.

         \begin{solution}[2.5in]   
        
         \end{solution}

         \part[1] When is Wilber's \emph{speed} zero?

         \begin{solution}[2.5in]   
        
         \end{solution}
    
    \end{parts}

\end{questions}

\vfill 
\noindent \textbf{Addendum} In Leibniz notation, it might seem peculiar that the derivative
order is a superscript of $\mathrm{d}$ in the numerator but a
superscript of the variable in the denominator. For example,
we write 
\(\displaystyle \frac{\mathrm{d}^3 y}{\mathrm{d}x^3}\) and
\textbf{ not } 
\(\displaystyle \frac{\mathrm{d}^3 y}{\mathrm{d}^3 x}\). 

This
notation might seem less peculiar if we were to define second and
higher order derivatives by a generalized Newton quotient. That 
would be a fun thing to do--let's save that for another day.

Another way to make the positioning of the derivative orders
seem less peculiar is to think about the physical units of 
the derivative. If we denote (special notation I just made up)
the physical unit of a quantity $q$ by $\langle q \rangle$, we 
have the rule
\[
    \left \langle \frac{\mathrm{d}^n y}{\mathrm{d} x^n} \right \rangle
    = \frac{ \langle y \rangle}{ \langle x \rangle^n }.
\]
The fact that the numerator of the units is \(\langle y \rangle\),
not \(\langle y \rangle^n\) and the denominator of the units is 
\(\langle x \rangle^n\), not \(\langle x \rangle\) makes the 
positioning of the superscripts seem more natural, I think.
Here is an example of this rule: if $\langle y \rangle = \mbox{ft}$ 
and \(\langle t \rangle = \mbox{sec}\), we have
\(\left \langle \frac{\mathrm{d}^2 y}{\mathrm{d} t^2} \right \rangle
    = \frac{ \mbox{ft}}{\mbox{sec}^2 }\).

Finally, for those of you taking CHEM 160: if you learn
how to simply multiply and divide physical quantities to
make the units come out correct, you'll be able to solve 
some multiple choice questions without understanding 
anything about chemistry. Using physical units is a
powerful way to check your work, but let's not substitute
it for a proper understanding.


\end{document}

