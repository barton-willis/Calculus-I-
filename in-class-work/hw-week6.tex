\documentclass[12pt,fleqn]{exam}
\usepackage{pifont}
\usepackage{dingbat}
\usepackage{amsmath,amssymb}
\usepackage{epsfig}
\usepackage{upgreek}
\usepackage[super]{nth}
\usepackage[colorlinks=true,linkcolor=black,anchorcolor=black,citecolor=black,filecolor=black,menucolor=black,runcolor=black,urlcolor=black]{hyperref}
\usepackage[letterpaper, margin=0.75in]{geometry}
\addpoints
\boxedpoints
\pointsinmargin
\pointname{pts}

\usepackage[activate={true,nocompatibility},final,tracking=true,kerning=true,factor=1100,stretch=10,shrink=10]{microtype}
\usepackage[american]{babel}
%\usepackage[T1]{fontenc}
\usepackage{fourier}
\usepackage{isomath}
\usepackage{upgreek,amsmath}
\usepackage{amssymb}
\usepackage{graphicx}

\newcommand{\dotprod}{\, {\scriptzcriptztyle\stackrel{\bullet}{{}}}\,}

\newcommand{\reals}{\mathbf{R}}
\newcommand{\lub}{\mathrm{lub}} 
\newcommand{\glb}{\mathrm{glb}} 
\newcommand{\complex}{\mathbf{C}}
\newcommand{\dom}{\mbox{dom}}
\newcommand{\range}{\mbox{range}}
\newcommand{\cover}{{\mathcal C}}
\newcommand{\integers}{\mathbf{Z}}
\newcommand{\vi}{\, \mathbf{i}}
\newcommand{\vj}{\, \mathbf{j}}
\newcommand{\vk}{\, \mathbf{k}}
\newcommand{\bi}{\, \mathbf{i}}
\newcommand{\bj}{\, \mathbf{j}}
\newcommand{\bk}{\, \mathbf{k}}
\DeclareMathOperator{\Arg}{\mathrm{Arg}}
\DeclareMathOperator{\Ln}{\mathrm{Ln}}
\newcommand{\imag}{\, \mathrm{i}}

\usepackage{graphicx}
\usepackage{color}
\shadedsolutions
\definecolor{SolutionColor}{rgb}{0.8,0.9,1}
\newcommand\AM{\textsc{am}}
\newcommand\PM{\textsc{pm}}
     
\newcommand{\quiz}{6}
\newcommand{\term}{Fall}
\newcommand{\due}{Wednesday 27 September 13:15 \PM}
\newcommand{\class}{MATH 115}
\begin{document}
\large
\vspace{0.1in}
\noindent\makebox[3.0truein][l]{\textbf{\class}}
\textbf{Name:} \hrulefill \\
\noindent \makebox[3.0truein][l]{\textbf{In class work \quiz, \term \/ \the\year}}
\textbf{Row and Seat}:\hrulefill\\
\vspace{0.1in}


\noindent  In class work  \quiz\/  has questions 1 through  \numquestions \/ with a total of  \numpoints\/  points.   
Turn in your work at the end of class  \emph{on paper}. This assignment is due \emph{\due}.

\vspace{0.1in}


\begin{questions} 

    \question As a function of time $t$ (seconds), the position $s$  (feet) 
    of a 1952 Farmall 44 horsepower tractor moving along a flat piece of Floyd's Creek Road 
    is given by $s = 2 t^{3/2}$. This relation holds
    for $1 \leq t \leq 9$.

    \begin{parts}

    \part [1] Find the \emph{displacement} of the tractor on the time
    interval $[1,9]$. That is, find $s \vert_{t=9} - s \vert_{t=1} $.

    \begin{solution}[1.5in]   
        
    \end{solution}

    \part [1] Find the \emph{average velocity} of the tractor on the time
    interval $[1,9]$. That is, find $\frac{s \vert_{t=9} - s \vert_{t=1}}{9-1} $.
    
    \begin{solution}[1.5in]   
        
    \end{solution}

    \part [1] Find the \emph{velocity} of the tractor at the time $t=1$.
    That is, find 
    $\displaystyle \left . \frac{\mathrm{d} s}{\mathrm{d} t} 
    \right \vert_{t=1}$.
    \begin{solution}[1.5in]   
        
    \end{solution}

    \part [1] Find the \emph{velocity} of the tractor at the time $t=9$.
    That is, find 
    $\displaystyle \left . \frac{\mathrm{d} s}{\mathrm{d} t} 
    \right \vert_{t=9}$.
    \begin{solution}[2.5in]   
        
    \end{solution}

    %\newpage 
    \part [1] Find the \emph{acceleration} of the tractor at the time $t=2$.
    That is, find 
    $\displaystyle \left . \frac{\mathrm{d} s}{\mathrm{d} t} 
    \right \vert_{t=9}$.
    \begin{solution}[2.5in]   
        
    \end{solution}

 

    \part [1] Show that $a v$, where $v$ is the velocity and $a$ is
    the acceleration of the tractor is a constant for times $t$ 
    in the interval $[1,9]$.
    \begin{solution}[2.5in]   
        
    \end{solution}
    
    \end{parts}

    \question The position $s$ of my pet American Fuzzy Lop rabbit Wilber moving along a line as a function of $t$ 
    is $s = \frac{1}{2} t^2 - 2t + 4$, where we consider positive values of $s$ to be
    to the right and negative to the left.

    \begin{parts}

         \part[1] When is Wilber moving to the right? That is, when is 
         $\displaystyle \frac{\mathrm{d} s}{\mathrm{d} t} > 0$?

         \begin{solution}[2.5in]   
        
         \end{solution}
         
         \part[1] When is Wilber moving to the left? That is, when is 
         $\displaystyle \frac{\mathrm{d} s}{\mathrm{d} t} < 0$?

         \begin{solution}[2.5in]   
        
         \end{solution}

         \part[1] Find Wilber's \emph{speed} when $t = 2$.

         \begin{solution}[2.5in]   
        
         \end{solution}
    
    \end{parts}

\end{questions}


\end{document}

