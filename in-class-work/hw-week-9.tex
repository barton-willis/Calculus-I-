\documentclass[12pt,fleqn]{exam}
\usepackage{pifont}
\usepackage{dingbat}
\usepackage{amsmath,amssymb}
\usepackage{epsfig}
\usepackage{upgreek}
\usepackage[super]{nth}
\usepackage[colorlinks=true,linkcolor=black,anchorcolor=black,citecolor=black,filecolor=black,menucolor=black,runcolor=black,urlcolor=black]{hyperref}
\usepackage[letterpaper, margin=0.75in]{geometry}
\addpoints
\boxedpoints
\pointsinmargin
\pointname{pts}
\usepackage{tikz}
\usepackage{tkz-euclide}
\usetikzlibrary{shapes.geometric}
\usetikzlibrary{calc}
\usepackage[activate={true,nocompatibility},final,tracking=true,kerning=true,factor=1100,stretch=10,shrink=10]{microtype}
\usepackage[american]{babel}
\usepackage[T1]{fontenc}
\usepackage[upright]{fourier}
\usepackage{isomath}
\usepackage{upgreek,amsmath}
\usepackage{amssymb}
\usepackage{graphicx}

\newcommand{\dotprod}{\, {\scriptzcriptztyle\stackrel{\bullet}{{}}}\,}

\newcommand{\reals}{\mathbf{R}}
\newcommand{\lub}{\mathrm{lub}} 
\newcommand{\glb}{\mathrm{glb}} 
\newcommand{\complex}{\mathbf{C}}
\newcommand{\dom}{\mbox{dom}}
\newcommand{\range}{\mbox{range}}
\newcommand{\cover}{{\mathcal C}}
\newcommand{\integers}{\mathbf{Z}}
\newcommand{\vi}{\, \mathbf{i}}
\newcommand{\vj}{\, \mathbf{j}}
\newcommand{\vk}{\, \mathbf{k}}
\newcommand{\bi}{\, \mathbf{i}}
\newcommand{\bj}{\, \mathbf{j}}
\newcommand{\bk}{\, \mathbf{k}}
\DeclareMathOperator{\Arg}{\mathrm{Arg}}
\DeclareMathOperator{\Ln}{\mathrm{Ln}}
\newcommand{\imag}{\, \mathrm{i}}

\usepackage{graphicx}
\usepackage{color}
\shadedsolutions
\definecolor{SolutionColor}{rgb}{0.8,0.9,1}
\newcommand\AM{\textsc{am}}
\newcommand\PM{\textsc{pm}}
     
\newcommand{\quiz}{8}
\newcommand{\term}{Fall}
\newcommand{\due}{Wednesday 20 October at 13:15 \PM}
\newcommand{\class}{MATH 115}
\begin{document}
\large
\vspace{0.1in}
\noindent\makebox[3.0truein][l]{\textbf{\class}}
\textbf{Name:} \hrulefill \\
\noindent \makebox[3.0truein][l]{\textbf{In class work \quiz, \term \/ \the\year}}
\textbf{Row and Seat}:\hrulefill\\
\vspace{0.1in}


\noindent  In class work  \quiz\/  has questions 1 through  \numquestions \/ with a total of  \numpoints\/  points.   
Turn in your work at the end of class  \emph{on paper}. This assignment is due \emph{\due}.

\vspace{0.1in}


\begin{questions} 

   \question  For the function $Q(x) = 2 {{x}^{3}}+3 {{x}^{2}}-36 x$, do the following:
   
   \begin{parts}
   
   \part [1] Find the location of all HTs. That is, solve $Q^\prime(x) = 0$.
   \begin{solution}[2.5in]
   
   \end{solution}
   
   \part [1] Find \(\underset{[-4,-2]}{\max} \,\,  Q $
   
     \begin{solution}[2.5in]
   
   \end{solution}
   
    \part [1] Find \(\underset{[-4,-2]}{\min} \,\,  Q $
   
     \begin{solution}[2.5in]
   
   \end{solution}
   
    
   
   \end{parts}
   
   \newpage
   
   
   \question For the function $J(x) = x^2/2  - \ln(x)$, do the following:
   
   \begin{parts}
   
    \part [1] Find the location of all HTs. That is, solve $J^\prime(x) = 0$.
   \begin{solution}[2.5in]
   
   \end{solution}
   
   \part [1] Find \(\underset{[1/2,2]}{\max} \,\,  J$
   
     \begin{solution}[2.5in]
   
   \end{solution}
   
    \part [1] Find \(\underset{[1/2,2]}{\min} \,\,  J $
   
  \end{parts}
   
\end{questions}


\end{document}

