\documentclass[12pt,fleqn,answers]{exam}
\usepackage{pifont}
\usepackage{dingbat}
\usepackage{amsmath,amssymb}
\usepackage{epsfig}
\usepackage[super]{nth}
\usepackage[colorlinks=true,linkcolor=black,anchorcolor=black,citecolor=black,filecolor=black,menucolor=black,runcolor=black,urlcolor=black]{hyperref}
\usepackage[letterpaper, margin=0.75in]{geometry}
\addpoints
\boxedpoints
\pointsinmargin
\pointname{pts}

\usepackage[activate={true,nocompatibility},final,tracking=true,kerning=true,factor=1100,stretch=10,shrink=10]{microtype}
\usepackage[american]{babel}
%\usepackage[T1]{fontenc}
\usepackage{fourier}
\usepackage{isomath}
\usepackage{upgreek,amsmath}
\usepackage{amssymb}
\usepackage{graphicx}

\newcommand{\dotprod}{\, {\scriptzcriptztyle
    \stackrel{\bullet}{{}}}\,}

\newcommand{\reals}{\mathbf{R}}
\newcommand{\lub}{\mathrm{lub}} 
\newcommand{\glb}{\mathrm{glb}} 
\newcommand{\complex}{\mathbf{C}}
\newcommand{\dom}{\mbox{dom}}
\newcommand{\range}{\mbox{range}}
\newcommand{\cover}{{\mathcal C}}
\newcommand{\integers}{\mathbf{Z}}
\newcommand{\vi}{\, \mathbf{i}}
\newcommand{\vj}{\, \mathbf{j}}
\newcommand{\vk}{\, \mathbf{k}}
\newcommand{\bi}{\, \mathbf{i}}
\newcommand{\bj}{\, \mathbf{j}}
\newcommand{\bk}{\, \mathbf{k}}
\DeclareMathOperator{\Arg}{\mathrm{Arg}}
\DeclareMathOperator{\Ln}{\mathrm{Ln}}
\newcommand{\imag}{\, \mathrm{i}}

\usepackage{graphicx}
\usepackage{color}
\shadedsolutions
\definecolor{SolutionColor}{rgb}{0.8,0.9,1}
\newcommand\AM{\textsc{am}}
\newcommand\PM{\textsc{pm}}
     
\newcommand{\quiz}{5}
\newcommand{\term}{Fall}
\newcommand{\due}{Wednesday 31 August at 13:15 \PM}
\newcommand{\class}{MATH 115}
\begin{document}
\large


\vspace{0.1in}


\begin{questions} 

\question Find the numerical value of each limit:

\begin{parts}

\part \(\displaystyle \lim_{x \to 9^{(-)}} \begin{cases} x+1 & x < 9 \\ \uppi &x \geq 9 \end{cases}\)
\part \(\displaystyle \lim_{x \to 9^{(+)}} \begin{cases} x+1 & x < 9 \\ \uppi &x \geq 9 \end{cases}\)
\part \(\displaystyle  \lim_{x \to 9}  \begin{cases} x+1 & x < 9 \\ \uppi &x \geq 9 \end{cases}\)
\part \(\displaystyle  \lim_{x \to 10}  \begin{cases} x+1 & x < 9 \\ \uppi &x \geq 9 \end{cases}\)
\part \(\displaystyle  \lim_{x \to -1}  \begin{cases} x+1 & x < 9 \\ \uppi &x \geq 9 \end{cases}\)
\part \(\displaystyle  \lim_{x \to 9} \frac{\sqrt{x+2} - \sqrt{11}}{x-9}\).
\part \(\displaystyle  \lim_{x \to 9} \frac{x^2 - 81}{x-9}\).

\end{parts}
\question Find each derivative

\begin{parts}

\part \(\displaystyle \frac{\mathrm{d}}{\mathrm{d} x } \left[ (x+9)(x^2 + 11) \right]\)
\part \(\displaystyle \frac{\mathrm{d}}{\mathrm{d} x } \left[ x (x+1)(x+2) \right]\)
\part \(\displaystyle \frac{\mathrm{d}}{\mathrm{d} x } \left[ \frac{x+9}{x^2 + 11}  \right]\)
\part \(\displaystyle \frac{\mathrm{d}}{\mathrm{d} x } \left[ \cos(5) x + \sin(32) \right]\)
\part \(\displaystyle \frac{\mathrm{d}}{\mathrm{d} x } \left[ \sqrt{100 x}  \right]\)

\end{parts}

\question Use a limit of a Newton quotient to show that the function $Q(x) = x^3 |x| + 8$ is differentiable at zero.

\question Find a TL to $y = x(x-8)$. The point of tangency is $(x=8, y = 0)$

\question Find the inverse to the function $K(x) = 5 x + 1$ and $\dom(K) = [-1,1]$.

\question Find the natural domain of  $K(x) = \frac{x+9}{8 - \frac{2}{x}} $.


\end{questions}
\end{document}