\documentclass[12pt,fleqn,answers]{exam}
\usepackage{pifont}
\usepackage{dingbat}
\usepackage{amsmath,amssymb}
\usepackage{epsfig}
\usepackage{upgreek}
\usepackage[super]{nth}
\usepackage[colorlinks=true,linkcolor=black,anchorcolor=black,citecolor=black,filecolor=black,menucolor=black,runcolor=black,urlcolor=black]{hyperref}
\usepackage[letterpaper, margin=0.5in]{geometry}
\addpoints
\boxedpoints
\pointsinmargin
\pointname{pts}
\usepackage{tikz}
\usepackage{tkz-euclide}
\usetikzlibrary{shapes.geometric}
\usetikzlibrary{calc}
\usepackage[activate={true,nocompatibility},final,tracking=true,kerning=true,factor=1100,stretch=10,shrink=10]{microtype}
\usepackage[american]{babel}
\usepackage[T1]{fontenc}
\usepackage[]{fourier}
\usepackage{isomath}
\usepackage{upgreek,amsmath}
\usepackage{amssymb}
\usepackage{graphicx}

\newcommand{\dotprod}{\, {\scriptzcriptztyle\stackrel{\bullet}{{}}}\,}

\newcommand{\reals}{\mathbf{R}}
\newcommand{\lub}{\mathrm{lub}} 
\newcommand{\glb}{\mathrm{glb}} 
\newcommand{\complex}{\mathbf{C}}
\newcommand{\dom}{\mbox{dom}}
\newcommand{\range}{\mbox{range}}
\newcommand{\cover}{{\mathcal C}}
\newcommand{\integers}{\mathbf{Z}}
\newcommand{\vi}{\, \mathbf{i}}
\newcommand{\vj}{\, \mathbf{j}}
\newcommand{\vk}{\, \mathbf{k}}
\newcommand{\bi}{\, \mathbf{i}}
\newcommand{\bj}{\, \mathbf{j}}
\newcommand{\bk}{\, \mathbf{k}}
\DeclareMathOperator{\Arg}{\mathrm{Arg}}
\DeclareMathOperator{\Ln}{\mathrm{Ln}}
\newcommand{\imag}{\, \mathrm{i}}

\usepackage{graphicx}
\usepackage{color}
\shadedsolutions
\definecolor{SolutionColor}{rgb}{0.8,0.9,1}
\newcommand\AM{\textsc{am}}
\newcommand\PM{\textsc{pm}}
     
\newcommand{\quiz}{12}
\newcommand{\term}{Fall}
\newcommand{\due}{Wednesday 9 November 13:15 \PM}
\newcommand{\class}{MATH 115}
\begin{document}
\large
\vspace{0.1in}
\noindent\makebox[3.0truein][l]{\textbf{\class}}
\textbf{Name:} \hrulefill \\
\noindent \makebox[3.0truein][l]{\textbf{In class work \quiz, \term \/ \the\year}}
\textbf{Row and Seat}:\hrulefill\\
\vspace{0.1in}


\noindent  In class work  \quiz\/  has questions 1 through  \numquestions \/ with a total of  \numpoints\/  points.   
Turn in your work at the end of class  \emph{on paper}. This assignment is due \emph{\due}.

\vspace{0.1in}


\begin{questions} 

    \question [5] Show that among all rectangles with a given perimeter,
    a square has the greatest area. 
    
    \quad To do this, let the lengths of the two perpendicular sides of the 
     rectangle be $x$ and $y$ and let $L$ be the perimeter of the
     rectangle. That makes $L = 2 x + 2 y$ a constraint. The other 
     constraints are $0 \leq x$ and $0 \leq y$.  We have $A = xy$.  Your task is to 
     maximize $A$ subject to the constraints $L = 2 x + 2 y, 0 \leq x$,
     and $0 \leq y$ with $L$  given.
     
     \begin{solution} Let's solve the constraint $L = 2 x + 2 y$  for $y$; the solution is $y =\frac{L - 2 x}{2}$. Pasting this into $A = xy$ gives 
     $A = x (\frac{L - 2 x}{2})$.  The graph of $A$ as a function of $x$ is a downward facing parabola that intersects the x-axis at $0$ and at $L/2$.
     The x-coordinate of the vertex of the parabola is halfway between the x-intercepts; thus the x-coordinate of the vertex is $x = L/4$.  Using
     $y =\frac{L - 2 x}{2}$, the y-coordinate of the vertex is $y = L/4$. 
     
     So to maximize the area of the rectangle, we need $x = L/4$ and $y = L/4$. And that's a square.
     
     \end{solution}

    %\newpage

    \question [5] Show that among all isosceles triangles with a given
    perimeter, the equilateral triangle has the greatest area.

    \quad To do this, let the lengths of the sides of the triangle be $x,x,$
    and $y$ and let $L$ be the perimeter. That makes $L = 2 x + y$ a
    constraint. The other constraints are $0 \leq x$ and $0 \leq y$.
    The area $A$ of the triangle is (this is a specialization of 
    the wonderful formula for the area of a triangle that is due to Hero of Alexandria 
    10 AD – c. 70 AD)
    \begin{equation*}
        16 {{A}^{2}}=4 {{x}^{2}}\, {{y}^{2}}-{{y}^{4}}.
    \end{equation*}
    Solve the constraint $L = 2 x + y$ for $y$ and paste that
    result in  the formula for $16 A^2$. Now do some calculus.
    \textbf{Hint:} Maximizing $16 A^2$ also maximizes $A$.  Thus
    alternatively, maximize the value of $Q$ where
    \begin{equation*}
        Q =4 {{x}^{2}}\, {{y}^{2}}-{{y}^{4}}.
    \end{equation*}
    You
    don't have to use this hint, but it's the easy way, I think.
    \quad If the algebra seems daunting to you, set $L = 3$ and work
    that specialization. 


   \begin{solution}
   Let's begin by solving the constraint  $L = 2 x + y$ for $y$; thus  $y = L - 2 x$.  
   But we also have $0 \leq x$ and $0 \leq y$.  But $y = L - 2 x$, so 
   $0 \leq L - 2x$. Put together, these inequalities tell us that
   $0 \leq x \leq L/2$. So we're looking at an optimization on a
   closed interval. 
   
   Paste $y = L - 2 x$ into $Q$. We have
   \begin{align*}
      Q  &=4 {{x}^{2}}\, {{y}^{2}}-{{y}^{4}},\\
          &= 4 {{\left( L-2 x\right) }^{2}}\, {{x}^{2}}-{{\left( L-2 x\right) }^{4}}, \\
          \intertext{We could either expand, factor, or LIB. We need to find the
          derivative of $Q$, so I think that LIB is a bad option.
          Expanding is tempting, but I see some opportunity to factor--let's factor.}
          &= L\, {{\left( 2 x-L\right) }^{2}}\, \left( 4 x-L\right).       
   \end{align*}
   This shows that $Q$ vanishes (and thus $A$ vanishes as well) when
   either $x = L/2$ or when $x = L/4$. Why does the area vanish when
   $x = L/2$. Oh, when $x = L/2$, we have $y = 0$. That gives a side
   length of zero, so the area is zero. A bit more 
   mysterious is why the area is zero when $x = L/4$.
   That makes $y = L/2$. Try drawing a triangle with
   side lengths of $L/4, L/4$, and $L/2$. What's
   the height?  It's zero.

   Now find the derivative of $Q$. We have
    \begin{align*}
      \frac{\mathrm{d} Q}{\mathrm{d} x}   &= 4 L\, {{\left( 2 x-L\right) }^{2}}+4 L\, \left( 2 x-L\right) \, \left( 4 x-L\right) ,\\
      \intertext{Again, we have a choice between LIB, factor, or expand. 
      Let's try the road less traveled and factor.}
                                                              &= 8 L\, \left( 2 x-L\right) \, \left( 3 x-L\right).
   \end{align*}   
   Now solve $\displaystyle \frac{\mathrm{d} Q}{\mathrm{d} x} = 0$. We have
   \begin{equation}
   \left[ 8 L\, \left( 2 x-L\right) \, \left( 3 x-L\right) = 0 \right] = \left[ x = \frac{L}{2}, x = \frac{L}{3} \right].
   \end{equation}
   So there are two CNs.  But if $x=  \frac{L}{2}$, then $y = L - L = 0$. Now that minimizes, not maximizes the area.  Actually, we have
   $0 \leq x$ and $0 \leq y$. But $y = L - 2 x$, so  $0 \leq L - 2 x$. So $0 \leq x \leq L/2$. 
  
  To find the maximum of $Q$, we need the chart:
   \begin{center}
\begin{tabular}{|c|c|} \hline 
    CN & Q \\ \hline \hline
    0  & 0 \\
    $L/3$ & $L^{4}/27$ \\
    $L/2$ & 0 \\ \hline
\end{tabular}
\end{center}
So the maximum area happens when $x = L/3$. And that makes $y = L/3$.
So the three side lengths are the same.
\end{solution}

   (this page is not quite blank)

\end{questions}


\end{document}

