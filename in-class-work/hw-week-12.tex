\documentclass[12pt,fleqn]{exam}
\usepackage{pifont}
\usepackage{dingbat}
\usepackage{amsmath,amssymb}
\usepackage{epsfig}
\usepackage{upgreek}
\usepackage[super]{nth}
\usepackage[colorlinks=true,linkcolor=black,anchorcolor=black,citecolor=black,filecolor=black,menucolor=black,runcolor=black,urlcolor=black]{hyperref}
\usepackage[letterpaper, margin=0.5in]{geometry}
\addpoints
\boxedpoints
\pointsinmargin
\pointname{pts}
\usepackage{tikz}
\usepackage{tkz-euclide}
\usetikzlibrary{shapes.geometric}
\usetikzlibrary{calc}
\usepackage[activate={true,nocompatibility},final,tracking=true,kerning=true,factor=1100,stretch=10,shrink=10]{microtype}
\usepackage[american]{babel}
\usepackage[T1]{fontenc}
\usepackage[]{fourier}
\usepackage{isomath}
\usepackage{upgreek,amsmath}
\usepackage{amssymb}
\usepackage{graphicx}

\newcommand{\dotprod}{\, {\scriptzcriptztyle\stackrel{\bullet}{{}}}\,}

\newcommand{\reals}{\mathbf{R}}
\newcommand{\lub}{\mathrm{lub}} 
\newcommand{\glb}{\mathrm{glb}} 
\newcommand{\complex}{\mathbf{C}}
\newcommand{\dom}{\mbox{dom}}
\newcommand{\range}{\mbox{range}}
\newcommand{\cover}{{\mathcal C}}
\newcommand{\integers}{\mathbf{Z}}
\newcommand{\vi}{\, \mathbf{i}}
\newcommand{\vj}{\, \mathbf{j}}
\newcommand{\vk}{\, \mathbf{k}}
\newcommand{\bi}{\, \mathbf{i}}
\newcommand{\bj}{\, \mathbf{j}}
\newcommand{\bk}{\, \mathbf{k}}
\DeclareMathOperator{\Arg}{\mathrm{Arg}}
\DeclareMathOperator{\Ln}{\mathrm{Ln}}
\newcommand{\imag}{\, \mathrm{i}}

\usepackage{graphicx}
\usepackage{color}
\shadedsolutions
\definecolor{SolutionColor}{rgb}{0.8,0.9,1}
\newcommand\AM{\textsc{am}}
\newcommand\PM{\textsc{pm}}
     
\newcommand{\quiz}{8}
\newcommand{\term}{Fall}
\newcommand{\due}{Wednesday 9 November 13:15 \PM}
\newcommand{\class}{MATH 115}
\begin{document}
\large
\vspace{0.1in}
\noindent\makebox[3.0truein][l]{\textbf{\class}}
\textbf{Name:} \hrulefill \\
\noindent \makebox[3.0truein][l]{\textbf{In class work \quiz, \term \/ \the\year}}
\textbf{Row and Seat}:\hrulefill\\
\vspace{0.1in}


\noindent  In class work  \quiz\/  has questions 1 through  \numquestions \/ with a total of  \numpoints\/  points.   
Turn in your work at the end of class  \emph{on paper}. This assignment is due \emph{\due}.

\vspace{0.1in}


\begin{questions} 

    \question [5] Show that among all rectangles with a given perimeter,
    a square has the greatest area. 
    
    \quad To do this, let the lengths of the two perpendicular sides of the 
     rectangle be $x$ and $y$ and let $L$ be the perimeter of the
     rectangle. That makes $L = 2 x + 2 y$ a constraint. The other 
     constraints are $0 \leq x$ and $0 \leq y$.  We have $A = xy$.  Your task is to 
     maximize $A$ subject to the constraints $L = 2 x + 2 y, 0 \leq x$,
     and $0 \leq y$ with $L$  given.

    \newpage

    \question Show that among all isosceles triangles with a given
    perimeter, the equilateral triangle has the greatest area.

    \quad To do this, let the lengths of the sides of the triangle be $x,x,$
    and $y$ and let $L$ be the perimeter. That makes $L = 2 x + y$ a
    constraint. The other constraints are $0 \leq x$ and $0 \leq y$.
    The area $A$ of the triangle is (this is a specialization of 
    the wonderful formula for the area of a triangle that is due to Hero of Alexandria 
    10 AD – c. 70 AD)
    \begin{equation*}
        A = (x+y) \sqrt{2 y (x + 2 y)}
    \end{equation*}
    Solve the constraint $L = 2 x + y$ for $y$ and paste that
    result in  the formula for $A$. Now do some calculus.
    \quad If the algebra seems daunting to you, set $L = 3$ and work
    that specialization. 


   
\end{questions}


\end{document}

