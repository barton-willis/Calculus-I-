\documentclass[12pt,fleqn,answers]{exam}
%\usepackage{pifont}
%\usepackage{dingbat,bbding}

\usepackage{amssymb}
\usepackage[intlimits]{amsmath}
\usepackage{epsfig}
\usepackage{upgreek}
\usepackage[super]{nth}
\usepackage[colorlinks=true,linkcolor=black,anchorcolor=black,citecolor=black,filecolor=black,menucolor=black,runcolor=black,urlcolor=black]{hyperref}
\usepackage[letterpaper, margin=0.75in]{geometry}
\addpoints
\boxedpoints
\pointsinmargin
\pointname{pts}
\usepackage{tikz}
\usepackage{tkz-euclide}
\usetikzlibrary{shapes.geometric}
\usetikzlibrary{calc}
\usepackage[activate={true,nocompatibility},final,tracking=true,kerning=true,factor=1100,stretch=10,shrink=10]{microtype}
\usepackage[american]{babel}
\usepackage[T1]{fontenc}
\usepackage[]{fourier}
\usepackage{isomath}
\usepackage{upgreek,amsmath}
\usepackage{amssymb}
\usepackage{graphicx}

\newcommand{\dotprod}{\, {\scriptzcriptztyle\stackrel{\bullet}{{}}}\,}

\newcommand{\reals}{\mathbf{R}}
\newcommand{\lub}{\mathrm{lub}} 
\newcommand{\glb}{\mathrm{glb}} 
\newcommand{\complex}{\mathbf{C}}
\newcommand{\dom}{\mbox{dom}}
\newcommand{\range}{\mbox{range}}
\newcommand{\cover}{{\mathcal C}}
\newcommand{\integers}{\mathbf{Z}}
\newcommand{\vi}{\, \mathbf{i}}
\newcommand{\vj}{\, \mathbf{j}}
\newcommand{\vk}{\, \mathbf{k}}
\newcommand{\bi}{\, \mathbf{i}}
\newcommand{\bj}{\, \mathbf{j}}
\newcommand{\bk}{\, \mathbf{k}}
\DeclareMathOperator{\Arg}{\mathrm{Arg}}
\DeclareMathOperator{\Ln}{\mathrm{Ln}}
\newcommand{\imag}{\, \mathrm{i}}

\usepackage{graphicx}
\usepackage{color}
\shadedsolutions
\definecolor{SolutionColor}{rgb}{1,0.72,0.46} %{0.8,0.9,1}
\newcommand\AM{\textsc{am}}
\newcommand\PM{\textsc{pm}}
     
\newcommand{\quiz}{13}
\newcommand{\term}{Fall}
\newcommand{\due}{Wednesday 16 November 13:15 \PM}
\newcommand{\class}{MATH 115}
\begin{document}
\large
\vspace{0.1in}
\noindent\makebox[3.0truein][l]{\textbf{\class}}
\textbf{Name:} \hrulefill \\
\noindent \makebox[3.0truein][l]{\textbf{In class work \quiz, \term \/ \the\year}}
\textbf{Row and Seat}:\hrulefill\\
\vspace{0.1in}


\noindent  In class work  \quiz\/  has questions 1 through  \numquestions \/ with a total of  \numpoints\/  points.   
Turn in your work at the end of class  \emph{on paper}. This assignment is due \emph{\due}.

\vspace{0.1in}


\begin{questions} 

\question Find a formula for each antiderivative.

\begin{parts}

    \part [1] $\int (6 x + 3)(x+1) \, \mathrm{d} x = $
    \begin{solution}[1.5in]
      The integrand is a product, not a sum. To use the additive property,
      our first step will be to expand the integrand. We have
     \begin{align*}
    \int (6 x + 3)(x+1) \, \mathrm{d} x &= \int 6 x^2 + 9 x + 3 \, \mathrm{d} x, &\mbox{(expand)}\\
                                        &= \int 6 x^2  \, \mathrm{d} x +
                                        \int 9 x \, \mathrm{d} x + 
                                        \int 3 \, \mathrm{d} x, &\mbox{(additive property)}\\
                                        &= 6 \int x^2  \, \mathrm{d} x +
                                        9 \int x \, \mathrm{d} x + 
                                        3 \int 1 \, \mathrm{d} x, &\mbox{(outative property)}\\
                                        &= 6 \times \frac{1}{3} x^3 +
                                           9 \times \frac{1}{2} x^2 +
                                           3 x + C, &\mbox{(power rule)} \\
                                        &= 2 {{x}^{3}}+\frac{9 {{x}^{2}}}{2}+3 x+C. &\mbox{(simplification)}
      \end{align*}
    Mathematical etiquette (ME)  \emph{requires} that we check every antiderivative 
    using a derivative. Maybe this polynomial result is sufficiently
    simple that it can be checked without a pencil, but here is the check:
    \begin{equation*}
      \frac{\mathrm{d}}{\mathrm{d} x} \left(2 {{x}^{3}}+\frac{9 {{x}^{2}}}{2}+3 x \right)
        = 6 x^2 + 9 x + 3 = (6 x + 3)(x+1). \,\,  \checkmark 
    \end{equation*}
    Since the derivative of a constant is zero, we never need
    to include the constant term in the check.
    \end{solution}
    
    \part [1] $\int (x-1)(x+2) \, \mathrm{d} x = $
    \begin{solution}[1.5in]
      Our process is much the same as for the first problem. 
      Here is the answer without the steps:
    \[
    \int (x-1)(x+2) \, \mathrm{d} x = \frac{{{x}^{3}}}{3}+\frac{{{x}^{2}}}{2}-2 x+C.
    \]
    ME dictates that we should check our work. I'll leave that for
    you to do.

    \quad Suppose your friend Pat says the answer is
    \begin{equation*}
      \frac{1}{6} \left( x-3\right) \, \left( 2 {{x}^{2}}+9 x+15\right) +C.
    \end{equation*}
    Who is correct? Let's expand Pat's answer and compare to ours. Expanding 
    yields
    \begin{equation*}
      \frac{{{x}^{3}}}{3}+\frac{{{x}^{2}}}{2}-2 x-\frac{15}{2} + C.
    \end{equation*}
    And that is different from our answer
    \begin{equation*}
      \frac{{{x}^{3}}}{3}+\frac{{{x}^{2}}}{2}-2 x+C.
    \end{equation*}
    Did we flub? Did Pat flub? Following proper ME, a simple check of 
    both answers show they are both correct. How can this be? Oh, it's 
    not such a great mystery: when they exist, antiderivatives are
    undetermined up to an additive constant.  Subtracting the two 
    answers yields a constant, so both are correct.
    \end{solution}

    \part [1] $\int \frac{7}{x} + \frac{x}{7} \, \mathrm{d} x = $
    \begin{solution}[1.5in]
    For the interval $(-\infty,0)$, we have
    \begin{align*}
    \int \frac{7}{x} + \frac{x}{7} \, \mathrm{d} x 
      &= \int \frac{7}{x} \, \mathrm{d} x  + \int \frac{x}{7} \, \mathrm{d} x, &\mbox{(additivity)} \\
      &= 7 \int \frac{1}{x} \, \mathrm{d} x  + \frac{1}{7} \int x \, \mathrm{d} x, &\mbox{(outative)} \\
      &= 7 \ln(-x) + \frac{x^2}{14} + C.
     \end{align*}
     Checking this we have
     \begin{equation*}
      \frac{\mathrm{d}}{\mathrm{d} x} \left( 7 \ln(-x) + \frac{x^2}{14} \right)
       = 7 \frac{-1}{-x} + \frac{x}{7}
       = \frac{7}{x} + \frac{x}{7} . \checkmark
     \end{equation*}
      For the interval $(0,\infty)$, we have
    \begin{equation*}
    \int \frac{7}{x} + \frac{x}{7} \, \mathrm{d} x = 7 \ln(x) + \frac{x^2}{14} + C.
     \end{equation*}

     Checking this we have
     \begin{equation*}
      \frac{\mathrm{d}}{\mathrm{d} x} \left( 7 \ln(x) + \frac{x^2}{14} \right)
       = 7 \frac{1}{x} + \frac{x}{7}
       = \frac{7}{x} + \frac{x}{7} . \checkmark
     \end{equation*}
     And for either the interval $(-\infty,0)$ or $(0,\infty)$, we have
      \begin{equation*}
    \int \frac{7}{x} + \frac{x}{7} \, \mathrm{d} x = 7 \ln(|x|) + \frac{x^2}{14} + C.
     \end{equation*}

     Checking this is tricky. to do the check we need
     the derivative of $\ln(|x|)$. Here it is:
     \begin{equation*}
      \frac{\mathrm{d}}{\mathrm{d} x} \left(\ln(|x|) \right)
     =  \frac{\mathrm{d}}{\mathrm{d} x}  \begin{cases}
        \ln(-x) & x < 0 \\
        \ln(x)  & x > 0 
     \end{cases}
     = 
     \begin{cases}
      \frac{-1}{-x} & x < 0 \\
      \frac{1}{x}  & x > 0 
    \end{cases}
    =
    \begin{cases}
      \frac{1}{x} & x < 0 \\
      \frac{1}{x}  & x > 0 
    \end{cases}
    = 
    \begin{cases}
      \frac{1}{x} & x \neq 0 
    \end{cases}
     \end{equation*}


     We do need to be careful with the result expressed this way.
     Since the natural logarithm function is not continuous at 
     zero, this antiderivative is not valid on any interval
     that contains zero. For example, the formula is valid
     on the interval $(-\infty,0)$ and it is valid on
     the interval $(0,\infty)$, but it is \textbf{invalid} 
     on the interval $(-\infty, \infty)$.
    \end{solution}

    \part [1] $\int \frac{x+1}{\sqrt{x}}  \, \mathrm{d} x = $
    \begin{solution}%[1.5in]
      \begin{align*}
        \int \frac{x+1}{\sqrt{x}}  \, \mathrm{d} x 
        &= \int \frac{x}{\sqrt{x}} + \frac{1}{\sqrt{x}} \, \mathrm{d} x, &\mbox{(expand)} \\
        &= \int x^{1/2} + x^{-1/2} \, \mathrm{d} x, &\mbox{(simplification)} \\
        &= \int x^{1/2} \, \mathrm{d} x + \int x^{-1/2} \, \mathrm{d} x, &\mbox{(additity)} \\
        &= \frac{2 {{x}^{\frac{3}{2}}}}{3}+2 \sqrt{x} + C.
      \end{align*}
    \end{solution}

%\newpage
    \part [1] $\int \cos(23 \uppi x)  \, \mathrm{d} x = $
    \begin{solution}[1.5in]
      For $a \in \reals_{\neq 0}$, we have the rule
      \begin{equation*}
         \int \cos(a x) \, \mathrm{d} x = \frac{1}{a} \sin(a x) + C.
      \end{equation*}
      Matching to this rule, we have
      \begin{equation*}
        \int \cos(23 \uppi x) \, \mathrm{d} x = 
        \frac{1}{23 \uppi} \sin(23 \uppi x) + C.
     \end{equation*}
    \end{solution}


    \part [1] $\int \cos(\uppi x)^2 +  \sin(\uppi x)^2 \, \mathrm{d} x = $
    \begin{solution}[1.5in]
      This one is a trick! The integrand simplifies to one. Thus
      \begin{equation*}
        \int \cos(\uppi x)^2 +  \sin(\uppi x)^2 \, \mathrm{d} x
        = \int 1 \, \mathrm{d} x = x + C.
      \end{equation*}
      If your first instinct was to use additivity, that is to use
      \begin{equation*}
        \int \cos(\uppi x)^2 +  \sin(\uppi x)^2 \, \mathrm{d} x
        = \int \cos(\uppi x)^2 \, \mathrm{d} x +  
        \int \sin(\uppi x)^2 \, \mathrm{d} x
            \end{equation*}\
    you were not alone. But taking this path, we're faced with
    evaluating 
    \begin{equation*}
      \int \cos(\uppi x)^2 \, \mathrm{d} x.
    \end{equation*}
    And we don't have the tools to do this. You might think
    the power rule for antiderivatives can be used on this 
    problem, but it cannot be used. If we match 
    \begin{equation*}
      \int x^n \, \mathrm{d} x \quad \mathrm{    to } \quad
      \int \cos(\uppi x)^2 \, \mathrm{d} x,
    \end{equation*}
    we first need to match $x^n$ to   
    $\cos(\uppi x)^2$. That's no problem; we simply 
    match $n \to 2$ and $x \to \cos(\uppi x)$. So far, so good.
    Are we done yet? No, we also need to match $\mathrm{d} x$ to 
    $\mathrm{d} x$. That match is easy: we simply match $x \to x$.
    But wait! We're already done the match $x \to \cos(\uppi x)$,
    so we can't also match $x \to  x$. Our attempt to use the 
    power rule failed--the attempted match was incomplete.

   
    \end{solution}
    
    \part [1] $ \int 5 \mathrm{d} x = $
    \begin{solution}[1.5in]
      \begin{equation*}
        \int 5 \mathrm{d} x  = \frac{5}{2} x^2 + C.
      \end{equation*}
      Yes, it really is that easy--it was intended to test your
      confidence. Problem sets are generally arranged from 
      easy to challenging--presented the other way, we can 
      be thrown off. But life isn't always arranged from
      easy to hard--we need to build our confidence.
    \end{solution}
    
    \part [1] $ \int \mathrm{e}^{5 x} \mathrm{d} x = $
    \begin{solution}[1.5in]
      For $a \in \reals_{\neq 0}$, we have the rule
    \begin{equation*}
      \int \mathrm{e}^{a x} \mathrm{d} x = \frac{1}{a} \mathrm{e}^{a x} + C.
    \end{equation*}
    Matching to this rule, we have
    \begin{equation*}
      \int \mathrm{e}^{5 x} \mathrm{d} x = \frac{1}{5} \mathrm{e}^{5 x} + C.
    \end{equation*}
    \end{solution}
\end{parts}

%\newpage

\question [1] Find numbers $a$ and $b$ such that
$
 \int x  \mathrm{e}^x \, \mathrm{d} x = (a + b x) \mathrm{e}^x + C
$
is correct. Do this by requiring that
$
\frac{\mathrm{d}}{\mathrm{d} x} \left((a + b x) \mathrm{e}^x \right)
   = x  \mathrm{e}^x
$ be an identity.
\begin{solution}
We need to find numbers $a$ and $b$ such that 
\begin{equation*}
  \frac{\mathrm{d}}{\mathrm{d} x}
  \left((a + b x) \mathrm{e}^x \right) = x  \mathrm{e}^x
\end{equation*}
is an identity. Evaluating the derivative on the left yields
\begin{equation*}
  b x\, {{\mathrm e}^{x}}+b\, {{\mathrm e}^{x}}+a\, {{\mathrm e}^{x}}=x\, {{\mathrm e}^{x}}.
\end{equation*}
Subtracting left form right and rearranging gives
\begin{equation*}
  \left( b-1\right)  x\, {{\mathrm e}^{x}}+\left( b+a\right) \, {{\mathrm e}^{x}}=0.
\end{equation*}
Dividing this by ${\mathrm e}^{x}$ (which is nonzero) gives
\begin{equation*}
  \left( b-1\right)  x +\left( b+a\right) =0
\end{equation*}
This is supposed to be an \emph{identity}. It is \textbf{not}
an equation that we are supposed to solve for $x$. Since 
it is supposed to be an identity, it means that for
\emph{every} real number $x$, we have $\left( b-1\right)  x +\left( b+a\right) =0$
How can this be? We \emph{must} choose the numbers $a$ and
$b$ such that $b-1= 0$ and $a+b=0$. Thus $b=1$ and $a=-1$.
Thus we are claiming that
$$
 \int x  \mathrm{e}^x \, \mathrm{d} x = (x - 1) \mathrm{e}^x + C.
$$
Checking this by differentiation shows that it is correct.

\quad The Risch integration method (\url{https://en.wikipedia.org/wiki/Risch_algorithm}) 
is something like this. Using a great deal of 
\end{solution}

\end{questions}

\end{document}

