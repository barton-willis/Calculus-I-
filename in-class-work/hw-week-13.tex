\documentclass[12pt,fleqn]{exam}
\usepackage{pifont}
\usepackage{dingbat}
\usepackage{amsmath,amssymb}
\usepackage{epsfig}
\usepackage{upgreek}
\usepackage[super]{nth}
\usepackage[colorlinks=true,linkcolor=black,anchorcolor=black,citecolor=black,filecolor=black,menucolor=black,runcolor=black,urlcolor=black]{hyperref}
\usepackage[letterpaper, margin=0.75in]{geometry}
\addpoints
\boxedpoints
\pointsinmargin
\pointname{pts}
\usepackage{tikz}
\usepackage{tkz-euclide}
\usetikzlibrary{shapes.geometric}
\usetikzlibrary{calc}
\usepackage[activate={true,nocompatibility},final,tracking=true,kerning=true,factor=1100,stretch=10,shrink=10]{microtype}
\usepackage[american]{babel}
\usepackage[T1]{fontenc}
\usepackage[]{fourier}
\usepackage{isomath}
\usepackage{upgreek,amsmath}
\usepackage{amssymb}
\usepackage{graphicx}

\newcommand{\dotprod}{\, {\scriptzcriptztyle\stackrel{\bullet}{{}}}\,}

\newcommand{\reals}{\mathbf{R}}
\newcommand{\lub}{\mathrm{lub}} 
\newcommand{\glb}{\mathrm{glb}} 
\newcommand{\complex}{\mathbf{C}}
\newcommand{\dom}{\mbox{dom}}
\newcommand{\range}{\mbox{range}}
\newcommand{\cover}{{\mathcal C}}
\newcommand{\integers}{\mathbf{Z}}
\newcommand{\vi}{\, \mathbf{i}}
\newcommand{\vj}{\, \mathbf{j}}
\newcommand{\vk}{\, \mathbf{k}}
\newcommand{\bi}{\, \mathbf{i}}
\newcommand{\bj}{\, \mathbf{j}}
\newcommand{\bk}{\, \mathbf{k}}
\DeclareMathOperator{\Arg}{\mathrm{Arg}}
\DeclareMathOperator{\Ln}{\mathrm{Ln}}
\newcommand{\imag}{\, \mathrm{i}}

\usepackage{graphicx}
\usepackage{color}
\shadedsolutions
\definecolor{SolutionColor}{rgb}{0.8,0.9,1}
\newcommand\AM{\textsc{am}}
\newcommand\PM{\textsc{pm}}
     
\newcommand{\quiz}{13}
\newcommand{\term}{Fall}
\newcommand{\due}{Wednesday 16 November 13:15 \PM}
\newcommand{\class}{MATH 115}
\begin{document}
\large
\vspace{0.1in}
\noindent\makebox[3.0truein][l]{\textbf{\class}}
\textbf{Name:} \hrulefill \\
\noindent \makebox[3.0truein][l]{\textbf{In class work \quiz, \term \/ \the\year}}
\textbf{Row and Seat}:\hrulefill\\
\vspace{0.1in}


\noindent  In class work  \quiz\/  has questions 1 through  \numquestions \/ with a total of  \numpoints\/  points.   
Turn in your work at the end of class  \emph{on paper}. This assignment is due \emph{\due}.

\vspace{0.1in}


\begin{questions} 

\question Find a formula for each antiderivative.

\begin{parts}

    \part [1] $\int (6 x + 3)(x+1) \, \mathrm{d} x = $
    \begin{solution}[1.5in]
     \begin{equation}
    \int (6 x + 3)(x+1) \, \mathrm{d} x = 2 {{x}^{3}}+\frac{9 {{x}^{2}}}{2}+3 x+C.
      \end{equation}
    \end{solution}
    
    \part [1] $\int (x-1)(x+2) \, \mathrm{d} x = $
    \begin{solution}[1.5in]
    \[
    \int (x-1)(x+2) \, \mathrm{d} x = \frac{{{x}^{3}}}{3}+\frac{{{x}^{2}}}{2}-2 x+C
    \]
    \end{solution}

    \part [1] $\int \frac{7}{x} + \frac{x}{7} \, \mathrm{d} x = $
    \begin{solution}[1.5in]
    For the interval $(-\infty,0)$, we have
    \begin{equation*}
    \int \frac{7}{x} + \frac{x}{7} \, \mathrm{d} x = 7 \ln(-x) + \frac{x^2}{14} + C.
     \end{equation*}
      For the interval $(0,\infty)$, we have
    \begin{equation*}
    \int \frac{7}{x} + \frac{x}{7} \, \mathrm{d} x = 7 \ln(x) + \frac{x^2}{14} + C.
     \end{equation*}
     And for either the interval $(-\infty,0)$ or $(0,\infty)$, we have
      \begin{equation*}
    \int \frac{7}{x} + \frac{x}{7} \, \mathrm{d} x = 7 \ln(|x|) + \frac{x^2}{14} + C.
     \end{equation*}
     
    \end{solution}

    \part [1] $\int \frac{x+1}{\sqrt{x}}  \, \mathrm{d} x = $
    \begin{solution}%[1.5in]
    \end{solution}

\newpage
    \part [1] $\int \cos(23 \uppi x)  \, \mathrm{d} x = $
    \begin{solution}[1.5in]
    \end{solution}


    \part [1] $\int \cos(\uppi x)^2 +  \sin(\uppi x)^2 \, \mathrm{d} x = $
    \begin{solution}[1.5in]
    \end{solution}
    
    \part [1] $ \int 5 \mathrm{d} x = $
    \begin{solution}[1.5in]
    \end{solution}
    
    \part [1] $ \int \mathrm{e}^{5 x} \mathrm{d} x = $
    \begin{solution}[1.5in]
    \end{solution}
\end{parts}

\newpage

\question [1] Find numbers $a$ and $b$ such that
$
 \int x  \mathrm{e}^x \, \mathrm{d} x = (a + b x) \mathrm{e}^x + C
$
is correct. Do this by requiring that
$
\frac{\mathrm{d}}{\mathrm{d} x} \left((a + b x) \mathrm{e}^x \right)
   = x  \mathrm{e}^x
$ be an identity.

\end{questions}

\end{document}

