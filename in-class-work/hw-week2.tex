\documentclass[12pt,fleqn]{exam}
\usepackage{pifont}
\usepackage{dingbat}
\usepackage{amsmath,amssymb}
\usepackage{epsfig}
\usepackage[super]{nth}
\usepackage[colorlinks=true,linkcolor=black,anchorcolor=black,citecolor=black,filecolor=black,menucolor=black,runcolor=black,urlcolor=black]{hyperref}
\usepackage[letterpaper, margin=0.75in]{geometry}
\addpoints
\boxedpoints
\pointsinmargin
\pointname{pts}

\usepackage[activate={true,nocompatibility},final,tracking=true,kerning=true,factor=1100,stretch=10,shrink=10]{microtype}
\usepackage[american]{babel}
%\usepackage[T1]{fontenc}
\usepackage{fourier}
\usepackage{isomath}
\usepackage{upgreek,amsmath}
\usepackage{amssymb}

\newcommand{\dotprod}{\, {\scriptzcriptztyle
    \stackrel{\bullet}{{}}}\,}

\newcommand{\reals}{\mathbf{R}}
\newcommand{\lub}{\mathrm{lub}} 
\newcommand{\glb}{\mathrm{glb}} 
\newcommand{\complex}{\mathbf{C}}
\newcommand{\dom}{\mbox{dom}}
\newcommand{\range}{\mbox{range}}
\newcommand{\cover}{{\mathcal C}}
\newcommand{\integers}{\mathbf{Z}}
\newcommand{\vi}{\, \mathbf{i}}
\newcommand{\vj}{\, \mathbf{j}}
\newcommand{\vk}{\, \mathbf{k}}
\newcommand{\bi}{\, \mathbf{i}}
\newcommand{\bj}{\, \mathbf{j}}
\newcommand{\bk}{\, \mathbf{k}}
\DeclareMathOperator{\Arg}{\mathrm{Arg}}
\DeclareMathOperator{\Ln}{\mathrm{Ln}}
\newcommand{\imag}{\, \mathrm{i}}

\usepackage{graphicx}
\newcommand\AM{{\sc am}}
\newcommand\PM{{\sc pm}}
     
\newcommand{\quiz}{2}
\newcommand{\term}{Fall}
\newcommand{\due}{Wednesday 31 August at 13:15 \PM}
\newcommand{\class}{MATH 115}
\begin{document}
\large
\vspace{0.1in}
\noindent\makebox[3.0truein][l]{\textbf{\class}}
\textbf{Name:} \hrulefill \\
\noindent \makebox[3.0truein][l]{\textbf{In class work \quiz, \term \/ \the\year}}
\textbf{Row and Seat}:\hrulefill\\
\vspace{0.1in}


\noindent  In class work  \quiz\/  has questions 1 through  \numquestions \/ with a total of  \numpoints\/  points.   
Turn in your work at the end of class  \emph{on paper}. This assignment is due \emph{\due}.

\vspace{0.1in}


\begin{questions} 

\question[5] After graduation, suppose your starting salary is \$46,000. Further, suppose
that you expect to earn a 4.1\% pay rise each year you work. What is your salary for 
your \nth{40} year of work?  \textbf{Hint:} Your salary for your \nth{3} year of work
is $\$46,000 \times 1.041^2$.
\begin{solution}[2.5in]

\end{solution}

\question[5] Let $Q(x) = \frac{6}{1+\exp(-x)}$. As best you can,
reproduce the  graph here.  Using the graph, find $\range(Q)$.
Be careful: Is zero in the range? What is the solution set to 
\mbox{$0 = \frac{6}{1+\exp(-x)}$?} Is six in the range? What is the solution set to 
\mbox{$6 = \frac{6}{1+\exp(-x)}$?}


\begin{solution}[3.5in]

\end{solution}

\newpage

\question[5] Define $Q(x) = (x-1)^2 + 1$ and $\dom(Q) = [1,\infty)$. Find the formula
and the domain of $Q^{-1}$. Use desmos to graph both $Q$ and $Q^{-1}$. As best you can,
reproduce your graphs here.
\end{questions}
\end{document}