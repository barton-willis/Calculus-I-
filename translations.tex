\documentclass[usenames,dvipsnames,fleqn]{beamer}
%\usetheme[height=7mm]{Rochester}
\usetheme{Boadilla} %{Rochester}
\setbeamertemplate{caption}[numbered]
\setbeamertemplate{footline}[text line]{%
\parbox{\linewidth}{\vspace*{-8pt}\hfill\insertshortauthor\hfill\insertpagenumber}}
\setbeamertemplate{navigation symbols}{}
\author[BW]{BW}
\usepackage{amsmath}\usepackage{amsthm}
\usepackage{bbding, isomath}
\usepackage{upgreek}
\usepackage{comment,isomath}
\usepackage{wrapfig}
%\usepackage{marvosym}
\usepackage{xcolor}
\usepackage{soul}
\usepackage{array}% http://ctan.org/pkg/array
\newcommand{\hlc}[2][yellow]{{\sethlcolor{#1}\hl{#2}}}
%\usepackage[pdftex]{graphicx}

\usepackage[english]{babel}
\usepackage[final]{microtype}
\usepackage[]{xcolor}
%\usefonttheme{professionalfonts}
%\usefonttheme{serif}

\newcommand{\reals}{\mathbf{R}}
\newcommand{\complex}{\mathbf{C}}
\newcommand{\integers}{\mathbf{Z}}
\DeclareMathOperator{\range}{range}
\DeclareMathOperator{\domain}{domain}
\DeclareMathOperator{\codomain}{codomain}
\DeclareMathOperator{\sspan}{span}
\DeclareMathOperator{\F}{F}
\DeclareMathOperator{\G}{G}
\DeclareMathOperator{\B}{B}
\DeclareMathOperator{\D}{D}
\DeclareMathOperator{\id}{id}
\DeclareMathOperator{\ball}{ball}
\DeclareMathOperator{\dom}{dom}
\newcommand{\nth}{$n^\text{th}$} 

%\definecolor{blue-violet}{rgb}{0.54, 0.17, 0.89}
%\definecolor{blue-violet}{rgb}{0.81,0.40,1.36}
%\definecolor{blue-violet}{RGB}{81, 40, 136}
\definecolor{blue-violet}{HTML}{004D86}
\definecolor{UNK-gold}{HTML}{E4A115}
\newenvironment{PenList}{
  \begin{enumerate}[\textcolor{UNK-gold}{\PencilRightDown}]
    \addtolength{\itemsep}{-1.0\itemsep}}
  {\end{enumerate}}
  
  

\usepackage{amsmath}
\DeclareMathOperator{\nullspace}{nullity}
\theoremstyle{definition}
\newtheorem{mydef}{Definition}
\newtheorem{myex}{Example}
\newtheorem{myth}{Theorem} 
\newtheorem{myfact}{Fact}
\newtheorem{metathm}{Meta Theorem}
\newtheorem{Question}{Question}
\newtheorem{Answer}{Qnswer}
\newtheorem{myproof}{Proof}
\newtheorem{hurestic}{Hurestic}

\DeclareMathOperator{\3F2}{{}_3  F_2}
\newcommand\pochhammer[2]{\left[\genfrac..{0pt}{}{#1}{#2}\right]}
\newmuskip\pFqmuskip

\newcommand*\pFq[6][8]{%
  \begingroup % only local assignments
  \pFqmuskip=#1mu\relax
  % make the comma math active
  % \mathcode`\,=\string"8000
  % and define it to be \pFqcomma
  \begingroup\lccode`\~=`\,
  \lowercase{\endgroup\let~}\pFqcomma
  % typeset the formula
      {}_{#2}\!\F_{#3}{\left[\genfrac..{0pt}{}{#4}{#5};#6\right]}   %\F{\left[\genfrac..{0pt}{}{#4}{#5};#6\right]}  %alt: {}_{#2}\F_{#3}{\left[\genfrac..{0pt}{}{#4}{#5};#6\right]}
  \endgroup
}

\newcommand{\pFqcomma}{\mskip \pFqmuskip}
\newcommand{\mydash}{\text{--}}
\usepackage{hyperref}
%------------------
\title[BT] % (optional, only for long titles)
{\textcolor{blue-violet}{\textbf{Taking derivatives to the \nth}} \\  \vspace{0.1in}   \textcolor{black}{Minisymposium} \\  \textcolor{black}{University of Nebraska at Kearney}}

\author[BW] % (optional, for multiple authors)
{Barton~Willis}
\institute[UNK] % (optional)
{
 % \inst{1}%
     Professor and Chair \\
     Department of Mathematics and Statistics\\
     {College of Arts and Sciences} \\
    University Nebraska at Kearney  \\
    Kearney, Nebraska 68849,  USA \\
   \href{mailto:willisb@unk.edu}{willisb@unk.edu} \\
    \vspace{0.2in}
    \today}
 
  \date{}

\usepackage{courier}
%\lstset{basicstyle=\ttfamily\footnotesize,breaklines=true}
%\lstset{framextopmargin=50pt,frame=bottomline}
%\usepackage{lipsum}  % generates filler text


%\begin{document}

\usepackage[]{graphicx}

\newcommand{\foo}{\hspace{-2.3pt}$\bullet$ \hspace{5pt}}

%--------
%usepackage[usenames,dvipsnames,svgnames,table]{color}

\usepackage{comment}
\usepackage{wrapfig}
\begin{document}

%\frame{\titlepage}

\begin{frame}\textcolor{blue-violet}{\Large Four simple rules (for translations)}

\vspace{0.1in}

Let \(E\) be any equation in the variables \(x\)  (horizontal) and \(y\)  (vertical). Let \(h,k\), and \(a\) be real numbers. 
\vspace{0.1in}
\begin{PenList}

\item  \(x \to x - h\) shifts the graph of \(E\) \emph{h units to the  right}.
\vspace{0.1in}
\item  \(y \to x - k\) shifts the graph of \(E\) \emph{k units up}.
\vspace{0.1in}
\item \(x \to x/a\) horizontally stretches the graph of \(E\) \emph{by a factor of a}.
\vspace{0.1in}
\item  \(y \to y/a\) vertically  stretches the graph of \(E\) \emph{by a factor of a}.

\vspace{0.1in}

\item It doesn't matter if the equation is solved for \(y\). 
\vspace{0.1in}

\item \(-h\)  units right is the same as \(h\) units left; \(-k\)  units up  is the same as \(k\) units down.  
\vspace{0.1in}

\item A stretch of a factor of \(a\) is the same as squeeze by a factor of \(1/a\).

\vspace{0.1in}

\item Oops, that's seven rules (but four of them are simple).

\end{PenList}


\vfill
\end{frame}

\end{document}

Let \(f\) and \(g\) be smooth functions.  For \(n \in \integers_{\geq 0}\), let \( f^{(n)} \) be the nth derivative of \(f\). Iterating the product rule gives (coefficients in boxes for emphasis)
\begin{align*}
(f g)^{(0)} &= {\color{red} \fbox{1}} f^{(0)} g^{(0)}, \\
 (f g)^{(1)} &=   {\color{red}\fbox{1} }  f^{(0)} g^{(1)} +  {\color{red} \fbox{1}}  f^{(1)} g^{(0)}, \\
  (f g)^{(2)} &= {\color{red} \fbox{1}}  f^{(0)} g^{(2)} +  {\color{red}\fbox{2}}   f^{(1)} g^{(1)} +  {\color{red} \fbox{1}} f^{(2)} g^{(0)}, \\
   (f g)^{(3)} &=   {\color{red} \fbox{1}} f^{(0)} g^{(3)} +  {\color{red} \fbox{3} } f^{(1)} g^{(2)} +  +  {\color{red} \fbox{3}}    f^{(2)} g^{(1)} +   {\color{red}\fbox{1}}  f^{(3)} g^{(0)} .
\end{align*}
Who else is  ready to declare victory?  

\vspace{0.1in}
\textbf{\textcolor{UNK-gold}{Proposition}} We have 
\[
   (f g)^{(n)}  =  \sum_{k=0}^n \binom{n}{k} f^{(k)} g^{(n-k)} ,  n \in \integers_{\geq 0}.
 \]
 \end{frame}
 \begin{comment}
 \begin{frame}{\textcolor{blue-violet}{\textbf{How to be dangerous}}}
 
 \textbf{Question:} If there were a such a thing as the order -1 derivative of a function, what would it be?

\vspace{0.5in}
 \textbf{Pretty good answer:} For the derivative operator \(\mathrm{D}\), we have \(\mathrm{D}^m \mathrm{D}^n =\mathrm{D}^{m+n} \), setting \(m \to 1\) and \(n \to -1\) gives
 \[
     \mathrm{D} \mathrm{D}^{(-1)} f = \mathrm{D}^{(0)} f = f
 \]
 Apparently,  \( \mathrm{D}^{(-1)} f  = \int f \).  And apparently
 \[
   \int fg =   (f g)^{(-1)}  =  \sum_{k=0}^{-1} \binom{-1}{k} f^{(k)} g^{(-1-k)}.
 \]
  \end{frame}

 
\begin{frame}{\textcolor{blue-violet}{\textbf{A short intro to the next 15 minutes of your life}}}

\textbf{Question:}  Is there an easy pattern? And if so, do we care?  (\(\mathrm{D} \) is the derivative operator)
\begin{align*}
\mathrm{D}^0 \sin &= \sin,  &  \mathrm{D}^0 \cos &= \cos, \\
\mathrm{D}^1 \sin &= \cos,  & \mathrm{D}^1 \cos &= -\sin,  \\
\mathrm{D}^2 \sin &= -\sin,  &  \mathrm{D}^2 \cos &=-\cos, \\
\mathrm{D}^3 \sin &= -\cos,  & \mathrm{D}^3 \cos &= \sin,  \\
\mathrm{D}^4 \sin &= \sin, &  \mathrm{D}^4 \cos &= \cos. 
\end{align*}

\textbf{Partial answer:}  For a nonnegative integer \(n\), we have
\[
    \mathrm{D}^n \sin (x)  = \sin \left(x  + \frac{n \pi}{2}  \right), \quad \mathrm{D}^n \cos (x)  = \cos \left(x  + \frac{n \pi}{2}  \right)
\]
Other than to annoy your calculus teacher, is this of any value?



\end{frame}

\begin{frame}{\textcolor{blue-violet}{\textbf{Being a better Math student:  Be rebellious}}}

\textbf{Partial answer:}  For a nonnegative integer \(n\), we have
\[
    \mathrm{D}^n \sin (x)  = \sin \left(x  + \frac{n \pi}{2}  \right), \quad \mathrm{D}^n \cos (x)  = \cos \left(x  + \frac{n \pi}{2}  \right).
\]

Might we lift the condition that \(n\)  is nonnegative?  How about negative \(n\)?  Indeed, we have
\begin{align*}
  \int \sin(x) \, dx  &= -\cos(x)  = \sin(x - \frac{\pi}{2})  = \mathrm{D}^{-1}  \sin (x) , \\
    \int \cos(x) \, dx  &= \sin(x)  = \cos(x - \frac{\pi}{2})  = \mathrm{D}^{-1}  \cos (x) .
\end{align*}


\begin{PenList}

\item Mathematics is full of opportunities to break rules, and attempt to extend results to new territory.
\end{PenList}

\end{frame}


\begin{frame}{A short table of long derivatives}
\setlength{\extrarowheight}{2pt}
\begin{tabular}{| c | c | c|}

\hline
  \(F(x) \)   &  \(F^{(n)}(x)  \)  &  condition(s) \\ \hline 
  \(\mathrm{e}^x \)   &  \(\mathrm{e}^x \)  & \(n \in \integers\) \\  \hline
   \(\cos(x)  \)   &  \(  \cos \left(x  + \frac{n \pi}{2}  \right)  \)  & \(n \in \integers\) \\ \hline
   \(\sin(x)  \)   &  \(  \sin \left(x  + \frac{n \pi}{2}  \right)  \)  & \(n \in \integers\) \\  \hline
   \(\tan(x) \)   &  Ask Francesco Fa\'a  di Bruno  & \(n \in \integers\) \\  \hline
   \(x^{\mu} \)  & \( (-1)^n (-\mu)_n  \, \, x^{\mu - n}  \)  &  \mbox{or}  \( \begin{cases}   n \in \integers_{\geq 0}  \land    \mu \in \complex  \\  n \in \integers \land \mu \notin \integers_{\geq 0}  \end{cases} \) \\  \hline
   f(x) g(x)     & \(\displaystyle  \sum_{k=0}^n \binom{n}{k} f^{(k)}(x) g^{(n-k)}(x) \) &   \(n \in \integers_{\geq 0}  \) \\  \hline
   
\end{tabular}
\begin{PenList}

\item  Francesco Fa\'a  di Bruno is know for his formula l for derivatives of composite functions and his work on  elliptic functions. Giuseppe Peano was his student.



 
\end{PenList}

\end{frame}

\end{comment}
\begin{frame}{Let's try our new tool}

For some \(\mu \in \complex\), let's try to find a simple formula for the n-th derivative of the function  
\[
   x  \in [-1,\infty) \mapsto  x^{3/2}  \sqrt{1+x},
\]  
 Let's define a sequence  \(\upphi\) by
 \begin{align*}
       \upphi_n &=  \mathrm{D}^{(n)} x^{3/2}  \sqrt{1+x},\\
             & =   \sum_{k=0}^n \binom{n}{k}  \left(\mathrm{D}^{(k)}  x^{3/2}  \right) \left( \mathrm {D}^{(n-k)}  \sqrt{1+x} \right). 
 \end{align*}
Each term  in the summand has nice recursion relation; for example
 \[
    \mathrm{D}^{(k+1)}  \sqrt{1+x} =  -  \frac{k - 1/2}{1+x}  \,\,  \mathrm{D}^{(k)}  \sqrt{1+x} .
 \]
 
\end{frame}

\begin{frame}{Sister(ly) advice from Sister Celine}

Mary Celine Fasenmyer (1906 -- 1996) is known for her work on hypergeometric functions and for her long career in higher Mercyhurst College. A age 40(!), she earned a doctorate from the  University of Michigan.
Her best known contrbutions to mathematics are in the paper ``A note on pure recurrence relations,''  \emph{The American Mathematical Monthly}, \textbf{56} (1949).  

\centering
\includegraphics[width=0.25\textwidth]{Sister}

Sister Celine's method tells us that for all nonnegative \(n\), we have
\[
%qqq(n+1)*(2*n*x+5*x+4*n+6)+2*qqq(n+2)*(x+1)+2*(n+1)^2*qqq(n)
2 (x+1) \upphi_{n+2} + (2 n x+5x+4n+6) \upphi_{n+1 } + 2(n+1)^2 \upphi_n = 0.
%  2 (1+x)(2+x) \upphi_{n+2} + (2 x + 3) (1 + 2n)  \upphi_{n+1} + 2 (n^2-1)  \upphi_{n} = 0.
\]
 \end{frame}
 \begin{frame}
 
 \textbf{Proposition} The only solution to the recursion
 \[
%qqq(n+1)*(2*n*x+5*x+4*n+6)+2*qqq(n+2)*(x+1)+2*(n+1)^2*qqq(n)
2 (x+1) \upphi_{n+2} + (2 n x+5x+4n+6) \upphi_{n+1 } + 2(n+1)^2 \upphi_n = 0.
%  2 (1+x)(2+x) \upphi_{n+2} + (2 x + 3) (1 + 2n)  \upphi_{n+1} + 2 (n^2-1)  \upphi_{n} = 0.
\]
that satisfies \(\upphi_{n+1} =\upphi_{n}^\prime \) for all integers \(n\) is
\[
    \upphi_{n} = (1/2)_n (1+x)^{1/2 - n}  \pFq{2}{1}{3/2, -n}{3/2-n}{x+1}.
\] 
Further, up to a constant
\[
   \int x^{3/2} \sqrt{1+x} \, \mathrm{d} x  =  \upphi_{-1}
\]

 \end{frame}
\end{document}